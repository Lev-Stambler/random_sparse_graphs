\section{DAG Label Obfuscation from Additive Overhead iO}
\subsection{DAG Randomized Traversal}
Say that we have a sparse, potentially exponentially sized, graph $\graph = (\verts, \edges)$ with polynomial depth $D$,
and forall $v \in \verts, \deg(v) \leq d$. Moreover, for simplicity,
assume that for all $v$, 
$$
\deg^{-1}(v) = \big|\set{u \in V \mid \exists j \in [d], \Gamma(u)_j = v}\big| \leq d.
$$
In words, there are at most $d$ edges into a vertex. As a note, our construction just requires
that $\deg^{-1}(\cdot) = O(1)$ but for the sake of simplicity we fix $\deg^{-1}(\cdot) \leq d$.

We also require that $\graph$ is equipped with a neighbor function, $\Gamma$, which can be computed in polynomial time.
% (TODO: padding).
We define a randomized and keyed labelling function $\phi: \binSet^\lambda \times \verts \rightarrow \binSet^{\poly(\lambda)}$ 
such that given, $\phi(K, v_0)$ for root $v_0$, a PPT adversary which runs in time at most $T(\lambda)$, $\advers$, which does not know a path from $v_0$ to $v$,
% TODO: defn extractor??
\begin{equation}
	\label{eq:guessPhi}
	\Pr[\advers(\mathcal{O}(\circNeigb), v_0, v, \labelFunc(K, v_0)) = \labelFunc(K, v)] \leq \epsilon
\end{equation}
for function $\circNeigb$ where $\circNeigb(\labelFunc(K, u)) = \labelFunc(K, \Gamma(u)_1), \dots, \labelFunc(K, \Gamma(u)_d)$ and the circuit is padded to size $\compDepth$.
if $\Gamma(u) \neq \emptyset$ and otherwise $\Gamma(u)$ returns a $\bot$ string.
We fix the adversary's advantage to $\epsilon < \poly(\lambda)$ and runtime to $T(\lambda) \leq \poly(\lambda, \frac{1}{\eps})$
as we will need to show
that a set of a potentially exponential number of games \emph{does not have exponential security loss}
nor or \emph{reduce down to security against an exponentially strong adversary}.

\subsection{Instantiation}
We define 
$
	\labelFunc(K, v) = F(K, v)
$ for $K \randomGet \set{0, 1}^\lambda$, and we can now define our neighbor function $\circNeigb$.
\begin{equation}
	\label{eq:neighbfn}
	\circNeigb(\labelFunc(K, v), v) = C_1\circ\cdots \circ C_\compDepth \circ \; \circNeigbBase(\labelFunc(K, v))
\end{equation}
where $\circNeigbBase$ is defined in Algorithm \ref{alg:neighb} and $C_1, \dots C_\compDepth$ is 
either the identity function or a simple circuit which 
will be defined within context. In any case, $C_1, \dots C_\compDepth$ are always the same fixed and $O(1)$ size.
We refer to $\compDepth$ as the ``composition depth'' of an $\HiO$ obfuscated circuit.
% TODO: smwhere else?
We will use the shorthand $\circNeigb$ to denote 
$ C_1 \circ \cdots \circ C_\compDepth \circ \; \HiO(\circNeigbBase)$.

\begin{algorithm}[H]
	\caption{
		The circuit for the neighbor function, $\circNeigbBase$.
	}
	\begin{algorithmic}[1]
		\Function{$\circNeigbBase$}{$X, v$}
			\If{$f(X) \neq f(F(K, v))$}
				\State \Return $\bot$
			\EndIf
		 	\If{$\Gamma(v) = \emptyset$}
				\State \Return $\bot$
			\EndIf
			\State $u_1, \dots u_d = \Gamma(v)$
			\State \Return $F(K, u_1), F(K, u_2), \dots, F(K, u_d)$
		\EndFunction
	\end{algorithmic}
	\label{alg:neighb}
\end{algorithm}

\begin{theorem}[Label Extractibility]
	\label{thm:labelExtract}
	Given an $\HiO$ scheme, then
	for all $v \in \verts$, and uniform fixed polynomial sized extractor $\extractor$ circuit, we have that if there
	exists a PPT adversary $\advers$ such that
	\begin{equation}
		\Pr[\advers(\HiO(\circNeigb), v_0, v, \labelFunc(K, v_0)) = \labelFunc(K, v)] > \epsilon		
	\end{equation}	
	then
	\begin{equation}
		\Pr[\extractor(\advers, \HiO(\circNeigb), v_0, v, \labelFunc(K, v_0)) = P] > \negl(\lambda)
	\end{equation}
	where $\epsilon$ is a fixed advantage such that $\epsilon < \poly(1 / \lambda)$,
	$P$ is a path from $v_0$ to $v$ in $\graph$, 
	and $S = O(\depth \cdot d)$.
\end{theorem}