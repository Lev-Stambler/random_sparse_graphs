\newcommand{\labelFunc}{\phi}
\newcommand{\imageFn}{\text{Image}}
\newcommand{\pathSuffix}{\text{Suff}}
\newcommand{\pathSet}{\mathcal{P}}
\newcommand{\obfFN}{\mathcal{O}}

\section{Using Weak Extractible Obfuscation}

\subsection{Graph Randomized Traversal}
Say that we have a sparse, potentially exponentially sized, graph $\graph = (\verts, \edges)$
and $\forall v \in \verts, \deg(v) = d$.
We also require that $\graph$ is equipped with a neighbor function, $\Gamma$, which can be computed in polynomial time.
% (TODO: padding).
We define a randomized and keyed labelling function $\phi: \binSet^\lambda \times \verts \rightarrow \binSet^{\poly(\lambda)}$ 
such that given, $\phi(K, v_0)$ for root $v_0$, an adversary, $\advers$, which does not know a path from $v_0$ to $v$,
\begin{equation}
	\label{eq:guessPhi}
	\Pr[\advers(\mathcal{O}(\circNeigb), v_0, v, \labelFunc(K, v_0)) = \labelFunc(K, v)] \leq \negl(\lambda)
\end{equation}
for function $\circNeigb$ where $\circNeigb(\labelFunc(K, u)) = \labelFunc(K, \Gamma(u)_1), \dots, \labelFunc(K, \Gamma(u)_d)$
if $\Gamma(u) \neq \emptyset$ and otherwise $\Gamma(u)$ returns a $\bot$ string;
and, $\mathcal{O}$ represents an indistinguishable obfuscator.

\subsection{Instantiation}
We define 
\begin{equation*}
	\labelFunc(K, v) = F(K, v).
\end{equation*}
For shorthand, we will write $\sigma_v$ to connote an attempted ``signature'' of $v$
where a correct signature is $F(K, v)$.

We can now define $\circNeigb$:
\begin{algorithm}[H]
	\caption{
		The circuit for the neighbor function, $\circNeigb$.
	}
	\begin{algorithmic}[1]
		\Function{$\circNeigb$}{$f(\sigma_v), v$}
			\If{$f(\sigma_v) \neq f(F(K, v))$}
				\State \Return $\bot$
			\EndIf
		 	\If{$\Gamma(v) = \emptyset$}
				\State \Return $\bot$
			\EndIf
			\State $u_1, \dots u_d = \Gamma(v)$
			\State \Return $f(F(K, u_1)), f(F(K, u_2)), \dots, f(F(K, u_d))$
		\EndFunction
	\end{algorithmic}
	\label{alg:neighb}
\end{algorithm}

We are going to show that \cref{eq:guessPhi} holds by first showing that
the non-existence of an extractor to find a path from $v_0$ to $v$ implies that $\advers$
necessarily does not know $\phi(K, c)$ for a $c \in C_V \subset V$ where the vertices in $C_V$ border 
a graph cut which separates $v_0$ and $v$. Then, we inductively build up a series of games to show that
$\advers$ cannot learn \emph{any} $\phi(K, v)$ for $v \in V_1$ where $V_1$ are the vertices on the right-hand side of the cut.

% TODO: define P way above
\begin{lemma}
	Assuming that there is no extractor $E$ such that $\Pr[E(\Gamma, v_0, v) = P] \geq \frac{1}{p(\lambda)}$
	where $P \in \pathSet$, then for any PPT $\advers$, there exists some graph cut 
	$C_E \subset E$ which separates $v_0$ and $v$ and a set $C_V$ such that
	\begin{equation}
		\label{eq:cutLabel}
		\Pr[\advers(\obfFN(\circNeigb), v_0, v, \phi(K, v_0)) \in \phi(K, C_V)] \leq \negl(\lambda).
	\end{equation}
		We define $C_V \subset V$ to be
	\begin{equation*}
		\set{ u \mid (w, u) \in C_E \text{ and } u \text { on the side of } v} \bigcup \set{ v \mid (w, u) \in C_E \text{ and } u \text { on the side of } v}.
	\end{equation*}
	In words, $C_V$ are the vertices just adjacent to the cut and on the same side as $v$.
	\begin{proof}
		We will show that if $\advers$ can break \cref{eq:cutLabel}, then we can construct an extractor,
		$E$, which finds a path from $v_0$ to $v$ with non-negligible probability.

		Assume that for every possible cut, $\advers$ is able to produce a single label in this cut for a vertex $w$.
		Then, we note that there must be at least 1 path from $v_0$ to $w$ and $v$ as otherwise, $w$ would not be in the cut.
		Moreover, we note that $\advers$ must be able to produce a label for all vertices on at least one path
		from $v_0$ to $w$ as otherwise, we can change the cut to include the edges between where
		$\advers$ is able to produce a label and not able to produce a label. Using the same argument,
		we can show that $\advers$ must be able to produce all labels on a path from $w$ to $v$.

		Note that $\advers$ is not given the specific cut $C_E$ but rather $C_E$ is chosen based off of the adversary.
		So, we can build an extractor to do the following:
		% TODO: is this selective security?????
		\begin{enumerate}
			\item Create an iO obfuscated circuit with a random key, $K'$, for $\circNeigb$ and create circuit $\obfFN(\circNeigb)$
			as well as $\phi(K', v_0)$
			\item Run $\advers(\obfFN(\circNeigb), v_0, v, \phi(K', v_0))$ to get all labels $\phi(K', v_0), \dots \phi(K', v)$
			for some path from $v_0$ to $v$.
			\item Recreate the path from $v_0$ to $v$ via checking which vertex matches to adjacent labels in the path:
			% TODO: this assumes no cycles!!!
			I.e.\ starting with $\ell = 0$, we can learn the $\ell + 1$ vertex via finding $j \in [d]$ such that
			$\circNeigb(\phi(K', v_\ell), v_\ell)_j \in \set{\phi(K', v_0), \dots, \phi(K', v)}$
			 and then setting $v_{\ell + 1} = \Gamma(v_\ell)_j$.
		\end{enumerate}
	\end{proof}
\end{lemma}

\begin{algorithm}[H]
	\caption{
		Circuit for the neighbor function, $\circNeigb^{w^*, \Gamma(w^*)_1, \dots, \Gamma(w^*)_d}$ with punctured PRF key
		$K(\set{w^*})$ and constant $z^*, z^*_1, z^*_2, \dots, z^*_d$
	}
	\begin{algorithmic}[1]
		\Function{$\circNeigb$}{$f(\sigma_v), v$}
			\If{$v \neq w$ and $f(\sigma_v) \neq f(F(K, v))$}
				\State \Return $\bot$
			\EndIf
			\If{$v = w$ and $f(\sigma_v) \neq z^*$}
				\State \Return $\bot$ 
			\EndIf
		 	\If{$\Gamma(v) = \emptyset$}
				\State \Return $\bot$
			\EndIf
			\If{$v = w$}
				\State \Return $z^*_1, z^*_2, \dots, z^*_d$
			\EndIf
			\State $u_1, \dots u_d = \Gamma(v)$
			\State \Return $f(F(K, u_1)), f(F(K, u_2)), \dots, f(F(K, u_d))$
		\EndFunction
	\end{algorithmic}
	\label{alg:neighbHyb1}
\end{algorithm}

 



	We are going to use a series of inductively built indistinguishable hybrids along with \cref{alg:neighbHyb1}
	to show that \cref{eq:guessPhi} holds.
	\begin{itemize}
		\item $\Hyb_0$: In the first hybrid, the following game is played
			\begin{enumerate}
				\item $K \randomGet \binSet^{\lambda'}$ and $\phi(K, v_0) = (F(K, v_0), v)$
				\item The challenger generates $\mathcal{O}(\circNeigb)$ and gives the program to $\advers$
				\item The challenger chooses a $v$ and gives the adversary $v$ in plaintext.
				\item $\advers$ outputs guess $g$ and wins if $g = \phi(K, v)$
			\end{enumerate}
		
		\item $\Hyb_1$: Let $\mathcal{P}$ be the set of all paths from $v_0$ to $v$. For each path $P \in \mathcal{P}$
		where $P$ is an ordered list of connected vertices, we have that the adversary does not know
		some part of $P$.
		We can note that this implies that $\advers$ does not know $\phi(K, p)$ for all $p \in P$ as then
		$\advers$ can recover $P$. Let $u_P$ be the first vertex in $P$ such that $\advers$ does not know 
		\emph{a path} from $v_0$ to $u_P$. Define $\pathSuffix'(P)$ to be the path in $P$ from this $u_P$ to $v$.
		Then, necessarily, $\advers$ does not know $\phi(K, w_P)$ for at least one $w_P \in \pathSuffix'(P)$
		as then $\advers$ would know a path from $v_0$ to $v$. Now, let $\pathSuffix(P)$ be the path which starts at $w_P$, ends at $v$.

		We now inductively build up a series of hybrids to show that a hybrid distribution
		which shows $\phi(K, s)$ for $s \in \pathSuffix(P)$ indistinguishable from random.
		We perform the following procedure for each $P \in \mathcal{P}$. So, for $P \in \mathcal{P}$,
		
		\begin{enumerate}
			\item For the base case, 
			let $U = \pathSuffix(P)_1$
			where $\pathSuffix(P)_1$ is the first vertex in $P$ such that $\advers$ does not know $\phi(K, p)$ for $p \in P$.
				\begin{enumerate}[label=(\alph*)]
					\item Set $w^* = p$. Then, replace $\circNeigb$ with $\circNeigb^{w^*, \Gamma(w^*)_1, \dots, \Gamma(w^*)_d}$ as defined in \cref{alg:neighbHyb1}.
					Fix the constant $z^* = f(F(K, p))$ and $z^*_1 = f(F(K, \Gamma(w^*)_1)), \dots, z^*_d = f(F(K, \Gamma(w^*)_d))$.
					\item Set $z^* = f(t), z^*_1 = f(t_1), \dots z^*_d = f(t_d)$ where $t, t_1, \dots, t_d$ are chosen at random %TODO: specify field size
					% TODO: we want to **in the game** have the advantage be for guessing **any** for the points in \mathcal{P}...
					% Then reduce advantage to breaking PRF security...
				\end{enumerate}
			% TODO: need to show that the adversary does not know \phi(K, w^u)' where this is a different encryption
			% of \phi(K, w^u) than one seen in a path
			\item For the $\ell$-th inductive step where $1 \leq \ell < |\pathSuffix(P)|$, we are going to assume that we are given a hybrid such that 
			$w^* = \pathSuffix(P)_{\ell}$ and $z^* = f(t), z^*_1 = f(t_1), \dots z^*_d = f(t_d)$ for random $t, \dots t_d$ in \cref{alg:neighbHyb1}.
			Now, we change the hybrid in a similar manner as in the base case:
			\begin{enumerate}[label=(\alph*)]
				% This hyb is indisting from the first one and thus holds (assuming the last ones held)
				\item Set $w^* = \pathSuffix(P)_{\ell + 1}$. Then, replace $\circNeigb$ with $\circNeigb^{w^*, \Gamma(w^*)_1, \dots, \Gamma(w^*)_d}$ as defined in \cref{alg:neighbHyb1}.
				Fix the constant $z^* = f(F(K, p))$ and $z^*_1 = f(F(K, \Gamma(w^*)_1)), \dots, z^*_d = f(F(K, \Gamma(w^*)_d))$.
				% This hyb holds by the inductive hypothesis + OWF security again
				% I.e. induc hypothesis says that we cannot forge a signature for any of the prior points
				% Then, we cannot forge this signature!!
				\item Set $z^* = f(t), z^*_1 = f(t_1), \dots z^*_d = f(t_d)$ where $t, t_1, \dots, t_d$ are chosen at random %TODO: specify field size
			\end{enumerate}
			to puncture on $\pathSuffix(P)_{\ell + 1}$ where we update $z^*, \dots z^*_d$ 
			with new randomness.
			
% TODO: knowing implies extracting implies hybrid BaseCase (2) holds by forging security!! (A cannot forge simply as that implies an extractor for (\phi(K, u), u)
% Knowing impl
		\end{enumerate}
	\end{itemize}
	Finally, we can note that if $\Hyb_0 \compInd \Hyb_1$,
	\begin{equation*}
		\Pr[\advers(\circNeigb, v_0, v, \phi(K, v_0)) = \phi(K, v)] 
		\compInd
		\Pr[\advers(\circNeigb^*, v_0, v, \phi(K, v_0)) = \phi(K, v)]
	\end{equation*}
	where $z^*$ in $\circNeigb^*$	is the image on a OWF of a randomly chosen point.
	Thus, if $\advers$ can produce $\phi(K, v) = (\sigma_v, v)$, then $\advers$
	can find a preimage for $z*$ under $f$ and thus break the security of a one way function.

	Now, we show that $\Hyb_0 \compInd \Hyb_1$.

	\begin{lemma}
		The game in $\Hyb_1 (1a)$ is indistinguishable from $\Hyb_0$.
		\begin{proof}
			As the functionality of $\circNeigb$ in $\Hyb_0$ equals that of $\Hyb_1 (1a)$,
			we have indistinguishable simply from the definition of indistinguishable obfuscation.
		\end{proof}
	\end{lemma}

	\begin{lemma}
		The game in $\Hyb_1 (1b)$ is indistinguishable from $\Hyb_1 (1a)$.
		\begin{proof}
			Here we argue that if the game in $\Hyb_1 (1b)$ is distinguishable from
			$\Hyb_1 (1a)$, then we can construct an adversary, $\adversB$, which can break the security of the PRF
			at the punctured point. 

			% TODO: wait is this trivial??? Like just use the extractor definition???
			% But then wait, something is not right here
			% Something 

			% TODO: change game
		\end{proof}
	\end{lemma}

	\begin{lemma}
		The game in $\Hyb_1 (2a)$ is indistinguishable from $\Hyb_0$ and, by the inductive hypothesis, all previous hybrids.
		\begin{proof}
			Again, we have that the circuit for $\circNeigb$ is the same in $\Hyb_0$ and $\Hyb_1 (2a)$.
			Thus, by the definition of indistinguishable obfuscation, these games are indistinguishable.
		\end{proof}
	\end{lemma}

	\begin{lemma}
		The game in $\Hyb_1 (2b)$ is indistinguishable from $\Hyb_1 (2a)$ and, by the inductive hypothesis, all previous hybrids.
		\begin{proof}
			TODO: PRF security + extractor part
		\end{proof}
	\end{lemma}

	