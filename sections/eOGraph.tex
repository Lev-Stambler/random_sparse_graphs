\newcommand{\labelFunc}{\phi}
\newcommand{\imageFn}{\text{Image}}
\newcommand{\pathSuffix}{\text{Suff}}

\section{Using Weak Extractible Obfuscation}

\subsection{Graph Randomized Traversal}
Say that we have a sparse, potentially exponentially sized, graph $\graph = (\verts, \edges)$
and $\forall v \in \verts, \deg(v) = d$.
We also require that $\graph$ is equipped with a neighbor function, $\Gamma$, which can be computed in polynomial time.
% (TODO: padding).
We define a randomized and keyed labelling function $\phi: \binSet^\lambda \times \verts \rightarrow \binSet^{\poly(\lambda)}$ 
such that given, $\phi(K, v_0)$ for root $v_0$, an adversary, $\advers$, which does not know a path from $v_0$ to $v$,
\begin{equation}
	\label{eq:guessPhi}
	\Pr[\advers(\mathcal{O}(\circNeigb), v_0, v, \labelFunc(K, v_0)) = \labelFunc(K, v)] \leq \negl(\lambda)
\end{equation}
for function $\circNeigb$ where $\circNeigb(\labelFunc(K, u)) = \labelFunc(K, \Gamma(u)_1), \dots, \labelFunc(K, \Gamma(u)_d)$
if $\Gamma(u) \neq \emptyset$ and otherwise $\Gamma(u)$ returns a $\bot$ string;
and, $\mathcal{O}$ represents an indistinguishable obfuscator.

\subsection{Instantiation}
We define 
\begin{equation*}
	\labelFunc(K, v) = (F(K, v), v).
\end{equation*}
For shorthand, we will write $\sigma_v$ to connote an attempted ``signature'' of $v$
where a correct signature is $F(K, v)$.

We can now define $\circNeigb$:
\begin{algorithm}[H]
	\caption{
		The circuit for the neighbor function, $\circNeigb$.
	}
	\begin{algorithmic}[1]
		\Function{$\circNeigb$}{$f(\sigma_v), v$}
			\If{$f(\sigma_v) \neq f(F(K, v))$}
				\State \Return $\bot$
			\EndIf
		 	\If{$\Gamma(v) = \emptyset$}
				\State \Return $\bot$
			\EndIf
			\State $u_1, \dots u_d = \Gamma(v)$
			\State \Return $f(F(K, u_1)), f(F(K, u_2)), \dots, f(F(K, u_d))$
		\EndFunction
	\end{algorithmic}
	\label{alg:neighb}
\end{algorithm}

\begin{algorithm}[H]
	\caption{
		Circuit for the neighbor function, $\circNeigb^{w^*, \Gamma(w^*)_1, \dots, \Gamma(w^*)_d}$ with punctured PRF key
		$K(\set{w^*})$ and constant $z^*, z^*_1, z^*_2, \dots, z^*_d$
	}
	\begin{algorithmic}[1]
		\Function{$\circNeigb$}{$f(\sigma_v), v$}
			\If{$v \neq w$ and $f(\sigma_v) \neq f(F(K, v))$}
				\State \Return $\bot$
			\EndIf
			\If{$v = w$ and $f(\sigma_v) \neq z^*$}
				\State \Return $\bot$ 
			\EndIf
		 	\If{$\Gamma(v) = \emptyset$}
				\State \Return $\bot$
			\EndIf
			\If{$v = w$}
				\State \Return $z^*_1, z^*_2, \dots, z^*_d$
			\EndIf
			\State $u_1, \dots u_d = \Gamma(v)$
			\State \Return $f(F(K, u_1)), f(F(K, u_2)), \dots, f(F(K, u_d))$
		\EndFunction
	\end{algorithmic}
	\label{alg:neighbHyb1}
\end{algorithm}

\begin{proof}[Proof of \cref{eq:guessPhi}]
	We are going to use a series of inductively built indistinguishable hybrids along with \cref{alg:neighbHyb1}
	to show that \cref{eq:guessPhi} holds.
	\begin{itemize}
		\item $\Hyb_0$: In the first hybrid, the following game is played
			\begin{enumerate}
				\item $K \randomGet \binSet^{\lambda'}$ and $\phi(K, v_0) = (F(K, v_0), v)$
				\item The challenger generates $\mathcal{O}(\circNeigb)$ and gives the program to $\advers$
				\item The challenger chooses a $v$ and gives the adversary $v$ in plaintext.
				\item $\advers$ outputs guess $g$ and wins if $g = \phi(K, v)$
			\end{enumerate}
		
		\item $\Hyb_1$: Let $\mathcal{P}$ be the set of all paths from $v_0$ to $v$. For each path $P \in \mathcal{P}$
		where $P$ is an ordered list of connected vertices, we have that the adversary does not know
		some part of $P$.
		We can note that this implies that $\advers$ does not know $\phi(K, p)$ for all $p \in P$ as then
		$\advers$ can recover $P$. Let $u_P$ be the first vertex in $P$ such that $\advers$ does not know 
		\emph{a path} from $v_0$ to $u_P$. Define $\pathSuffix'(P)$ to be the path in $P$ from this $u_P$ to $v$.
		Then, necessarily, $\advers$ does not know $\phi(K, w_P)$ for at least one $w_P \in \pathSuffix'(P)$
		as then $\advers$ would know a path from $v_0$ to $v$. Now, let $\pathSuffix(P)$ be the path which starts at $w_P$, ends at $v$.

		We now inductively build up a series of hybrids to show that a hybrid distribution
		which shows $\phi(K, s)$ for $s \in \pathSuffix(P)$ indistinguishable from random.
		We perform the following procedure for each $P \in \mathcal{P}$. So, for $P \in \mathcal{P}$,
		
		\begin{itemize}
			\item For the base case, 
			let $U = \pathSuffix(P)_1$
			where $\pathSuffix(P)_1$ is the first vertex in $P$ such that $\advers$ does not know $\phi(K, p)$ for $p \in P$.
				\begin{enumerate}
					\item Set $w^* = p$. Then, replace $\circNeigb$ with $\circNeigb^{w^*, \Gamma(w^*)_1, \dots, \Gamma(w^*)_d}$ as defined in \cref{alg:neighbHyb1}.
					Fix the constant $z^* = f(F(K, p))$ and $z^*_1 = f(F(K, \Gamma(w^*)_1)), \dots, z^*_d = f(F(K, \Gamma(w^*)_d))$.
					\item Set $z^* = f(t), z^*_1 = f(t_1), \dots z^*_d = f(t_d)$ where $t, t_1, \dots, t_d$ are chosen at random %TODO: specify field size
					% TODO: we want to **in the game** have the advantage be for guessing **any** for the points in \mathcal{P}...
					% Then reduce advantage to breaking PRF security...
				\end{enumerate}
			% TODO: need to show that the adversary does not know \phi(K, w^u)' where this is a different encryption
			% of \phi(K, w^u) than one seen in a path
			\item For the $\ell$-th inductive step where $1 \leq \ell < |\pathSuffix(P)|$, we are going to assume that we are given a hybrid such that 
			$w^* = \pathSuffix(P)_{\ell}$ and $z^* = f(t), z^*_1 = f(t_1), \dots z^*_d = f(t_d)$ for random $t, \dots t_d$ in \cref{alg:neighbHyb1}.
			Now, we change the hybrid in a similar manner as in the base case: \begin{enumerate}
				% This hyb is indisting from the first one and thus holds (assuming the last ones held)
				\item Set $w^* = \pathSuffix(P)_{\ell + 1}$. Then, replace $\circNeigb$ with $\circNeigb^{w^*, \Gamma(w^*)_1, \dots, \Gamma(w^*)_d}$ as defined in \cref{alg:neighbHyb1}.
				Fix the constant $z^* = f(F(K, p))$ and $z^*_1 = f(F(K, \Gamma(w^*)_1)), \dots, z^*_d = f(F(K, \Gamma(w^*)_d))$.
				% This hyb holds by the inductive hypothesis + OWF security again
				% I.e. induc hypothesis says that we cannot forge a signature for any of the prior points
				% Then, we cannot forge this signature!!
				\item Set $z^* = f(t), z^*_1 = f(t_1), \dots z^*_d = f(t_d)$ where $t, t_1, \dots, t_d$ are chosen at random %TODO: specify field size
			\end{enumerate}
			to puncture on $\pathSuffix(P)_{\ell + 1}$ where we update $z^*, \dots z^*_d$ 
			with new randomness.
			
% TODO: knowing implies extracting implies hybrid BaseCase (2) holds by forging security!! (A cannot forge simply as that implies an extractor for (\phi(K, u), u)
% Knowing impl
		\end{itemize}
	\end{itemize}
	Finally, we can note that if $\Hyb_0 \compInd \Hyb_3$,
	\begin{equation*}
		\Pr[\advers(\circNeigb, v_0, v, \phi(K, v_0)) \in \imageFn(\phi(K, v))] 
		\compInd
		\Pr[\advers(\circNeigb', v_0, v, \phi(K, v_0)) \in \imageFn(\phi(K, v))] 
	\end{equation*}
	where $\circNeigb'$ is $\circNeigb$ except that $\circNeigb'$ uses $\inner_i^p$ where $p = \max_{P \in \mathcal{P}} |P|$.
	We can note that $\circNeigb'$ returns $\bot$ for any query on $\phi(K, w^v)$ where $w^v \in \Gamma^{-1}(v)$.
	Using \cref{lem:NMIndepA} and the fact that $\circNeigb'(u)_i$ returns $\bot$ for all $u \in \verts$ and $i \in [d]$ where $v = \Gamma(u)_i$, we have that
	$$
		\Pr[\advers(\circNeigb', v_0, v, \phi(K, v_0)) \in \imageFn(\phi(K, v))] \leq \negl(\lambda).
	$$
	
\end{proof}

\begin{lemma}
	\label{lem:hybrOneToFour}
	$\Hyb_0 \compInd \Hyb_{2b}$.	
	\begin{proof}
		First we show that $\Hyb_0 \compInd \Hyb_1$. Note that if $\advers$ can distinguish
		between $\Hyb_0$ and $\Hyb_1$ then an adversary can distinguish between an FE scheme and its simulated counterpart
		where $m$ is fixed to $(K, v_0, r)$. We can see this as $\Hyb_1$ is  direct simulation of the FE scheme.

		Then, if $\advers$ can distinguish $\Hyb_1$ and $\Hyb_{2a}$, then we can break the security of the $\PRG$ used in line \ref{alg:neighb:prg} of \cref{alg:neighb}.
		We can create an adversary $\adversB$ which, for some fixed $K$, distinguishes between $\FEEnc(\MPK, (K, u, r_2))$ with random coins $r_1$ where $r_1, r_2 = \PRG(r)$
		and $\FEEnc(\MPK, (K, u, r_1^*))$ encrypted with random coins $r_2^*$ where $r_1^*, r_2^*$ are truly random.

		Then, if $\advers$ can distinguish any transformation from $\Hyb_{2a}$ to $\Hyb_{2b}$, then we can break the security of the FE scheme.
		We can see this by noting that if we fix $m = (K, w, r)$ for random $r$ and $K$,
		then $\advers^{\Sim_3^{U_m(\cdot)}}(\FECT)$ is distinguishable
		and $\advers^{\Sim_3^{u_m(\cdot)}}(\FECT')$ where $\FECT$ is the real cipher-text and $\FECT'$ is simulated.
		We can then note that if the above are distinguishable, then
		$\advers^{\text{KeyGen}(\MSK, \set{\inner_1, \dots \inner_d})}(\FECT)$ 
		and $\advers^{\Sim_3^{u_m(\cdot)}}(\FECT')$ are distinguishable
		as $\advers^{\text{KeyGen}(\MSK, \set{\inner_1, \dots \inner_d})}$ can simply simulate
		$\advers^{\Sim_3^{U_m(\cdot)}}(\FECT)$.

		Then, if $\advers$ can distinguish any transformation from $\Hyb_{2b}$ to $\Hyb_{2a}$, then we can break the security of a PRG
		in the same manner as distinguishing $\Hyb_1$ and $\Hyb_{2a}$.

		By the chain rule, we get that $\Hyb_0$ and $\Hyb_{2b}$ are indistinguishable even after a repeated
		number of sequential invocations of the transformation in $\Hyb_{2a}$ and $\Hyb_{2b}$.
	\end{proof}
\end{lemma}


\begin{lemma}
		\label{lem:NMIndep}
		Let $\advers$ be a PPT adversary and assume that we have a non-malleable and simulation secure
		FE scheme. Then, we have that the inductive step of $\Hyb_3$ holds.
		\begin{proof}
			We construct an adversary $\adversB$ that can break NM security using $\advers$ if $\advers$ can distinguish between the hybrids
			in the inductive step.
			Note that in order to distinguish between the hybrids, $\advers$ must have queried $\inner^\ell_i$ or $\inner^\ell_{i + 1}$ on $\phi(K, w^u)$ where $u \in \set{\pathSuffix(P)_{\ell + 1} \mid P \in \mathcal{P}}$
			as this is the only difference between the hybrids.
			Thus, we see that $\advers$ is able to produce $\FECT \in \phi(K, w^u)$.
			By definition of $\inner_i^\ell$ though, we know that $\inner_i^\ell(\phi(k, q)) \neq \phi(K, w^u)$
			for any $q \in \verts$ as we define $\inner_i^\ell(K, q) = \bot$ if $\inner_i'(K, q) = \phi(K, w^u)$.
			Thus, the adversary has to be able to produce $\FECT \in \phi(K, w^u)$ without calling $\circNeigb^\ell$
			where $\circNeigb^\ell$ uses $\inner_i^\ell$ instead of $\inner_i$.

			Thus, if $\advers(w^u, v_0, \circNeigb, \phi(K, v_0))$ can produce $\FECT \in \phi(K, w^u)$,
			we can have $\adversB(\phi(K, v_0), \phi(K, q_1), \dots,$ $ \phi(K,q_{\poly(\lambda)}))$ 
			produce $\phi(K, w^u)$ where $q_1, \dots, q_{\poly(\lambda)}$ are all the vertices that $\advers$ has queried $\circNeigb$ on.
			$\adversB$ simply has to invoke $\Sim_3$ to create a simulated function key for $\SK_{\inner_i}'$
			and thus a simulated $\circNeigb'$.
			$\adversB$ then gives $\advers$ $(w^u, v_0, \circNeigb', \phi(K, v_0))$.
			$\adversB$ then breaks \cref{eq:NMRel} (the relational notion of non-malleability) as $\advers$ is able to create an encryption of $\phi(K, w^u)$ with non-negligible probability while the simulator 
			in \cref{eq:NMRel} cannot.
		\end{proof}
\end{lemma}

\begin{lemma}
		\label{lem:NMIndepA}
		Define $\circNeigb'$ where $\circNeigb'$ is defined as in \cref{alg:neighb} except that
		for some set $U \subset \verts$, $\circNeigb(w^u)_i = \bot$ for all $w^u \in \verts$ such that $u = \Gamma(w^u)_i$ for some $u \in U$.
		In words, the parent of all $u \in U$ do not return $\phi(K, u)$ when queried on $\circNeigb'$.
		Then, assuming the non-malleability and simulation security of FE, we have that for all PPT $\advers$
		and all $u \in U$,
		\begin{equation}
			\Pr[\advers(\circNeigb', v_0, u, U, \phi(K, v_0)) \in \imageFn(\phi(K, u))] \leq \negl(\lambda).
		\end{equation}
		\begin{proof}
			Almost identically to \cref{lem:NMIndep}, we construct an adversary $\adversB$ that can break NM security using $\advers$ if $\advers$ can
			produce $\FECT \in \phi(K, u)$ for some $u \in U$.
			
			If $\advers(w^u, v_0, \circNeigb', u, \phi(K, v_0))$ can produce $\FECT \in \phi(K, u)$,
			we can have $\adversB(\phi(K, v_0), \phi(K, q_1), \dots,$ $ \phi(K,q_{\poly(\lambda)}))$ 
			produce $\phi(K, u)$ where $q_1, \dots, q_{\poly(\lambda)}$ are all the vertices that $\advers$ has queried $\circNeigb'$ on.
			$\adversB$ simply has to invoke $\Sim_3$ to create a simulated set of function keys for $\inner_i'$ for all $i \in [d]$
			and can then simulate $\circNeigb'$ with these function keys.

			We can then have $\adversB$ invoke $\Sim_3$ to create a simulated function key for $\SK_{\inner_i}'$
			and thus a simulated $\circNeigb^{*}$.
			$\adversB$ then gives $\advers$ $(w^u, v_0, \circNeigb^{*}, \phi(K, v_0))$.
			If we define the relation $R$ to break in \cref{eq:NMRel} to be $R(K, v_0, r) = \set{(K, v, r*) : \forall r^* \gets \set{0, 1}^\lambda}$,
			we can then break \cref{eq:NMRel} (the relational notion of security for non-malleability).
			We can see this as $\advers$ is able to create an encryption of $\phi(K, w^u)$
			given encryptions of $\phi(K, q_1), \dots, \phi(K, q_{\poly(\lambda)})$
			with non-negligible probability while the simulator 
			in \cref{eq:NMRel} cannot.
		\end{proof}
\end{lemma}

