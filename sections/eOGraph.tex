\newcommand{\labelFunc}{\phi}
\newcommand{\imageFn}{\text{Image}}
\newcommand{\pathSuffix}{\text{Suff}}
\newcommand{\pathSet}{\mathcal{P}}
\newcommand{\obfFN}{\mathcal{O}}

\section{Using Weak Extractible Obfuscation}

\subsection{Graph Randomized Traversal}
Say that we have a sparse, potentially exponentially sized, graph $\graph = (\verts, \edges)$
and $\forall v \in \verts, \deg(v) = d$.
We also require that $\graph$ is equipped with a neighbor function, $\Gamma$, which can be computed in polynomial time.
% (TODO: padding).
We define a randomized and keyed labelling function $\phi: \binSet^\lambda \times \verts \rightarrow \binSet^{\poly(\lambda)}$ 
such that given, $\phi(K, v_0)$ for root $v_0$, an adversary, $\advers$, which does not know a path from $v_0$ to $v$,
\begin{equation}
	\label{eq:guessPhi}
	\Pr[\advers(\mathcal{O}(\circNeigb), v_0, v, \labelFunc(K, v_0)) = \labelFunc(K, v)] \leq \negl(\lambda)
\end{equation}
for function $\circNeigb$ where $\circNeigb(\labelFunc(K, u)) = \labelFunc(K, \Gamma(u)_1), \dots, \labelFunc(K, \Gamma(u)_d)$
if $\Gamma(u) \neq \emptyset$ and otherwise $\Gamma(u)$ returns a $\bot$ string;
and, $\mathcal{O}$ represents an indistinguishable obfuscator.

\subsection{Instantiation}
We define 
\begin{equation*}
	\labelFunc(K, v) = F(K, v).
\end{equation*}
For shorthand, we will write $\sigma_v$ to connote an attempted ``signature'' of $v$
where a correct signature is $F(K, v)$.

We can now define $\circNeigb$:
\begin{algorithm}[H]
	\caption{
		The circuit for the neighbor function, $\circNeigb$.
	}
	\begin{algorithmic}[1]
		\Function{$\circNeigb$}{$f(\sigma_v), v$}
			\If{$f(\sigma_v) \neq f(F(K, v))$}
				\State \Return $\bot$
			\EndIf
		 	\If{$\Gamma(v) = \emptyset$}
				\State \Return $\bot$
			\EndIf
			\State $u_1, \dots u_d = \Gamma(v)$
			\State \Return $f(F(K, u_1)), f(F(K, u_2)), \dots, f(F(K, u_d))$
		\EndFunction
	\end{algorithmic}
	\label{alg:neighb}
\end{algorithm}

We are going to show that \cref{eq:guessPhi} holds by first showing that
the non-existence of an extractor to find a path from $v_0$ to $v$ implies that $\advers$
necessarily does not know $\phi(K, c)$ for a $c \in C_V \subset V$ where the vertices in $C_V$ border 
a graph cut which separates $v_0$ and $v$. Then, we inductively build up a series of games to show that
$\advers$ cannot learn \emph{any} $\phi(K, v)$ for $v \in V_1$ where $V_1$ are the vertices on the right-hand side of the cut.

% TODO: define P way above
\begin{lemma}[Base Case Game]
	\label{lemma:cutBaseCase}
	Assuming that there is no extractor $E$ such that $\Pr[E(\Gamma, v_0, v) = P] \geq \frac{1}{p(\lambda)}$
	where $P \in \pathSet$, then for any PPT $\advers$, there exists some graph cut 
	$C_E \subset E$ which separates $v_0$ and $v$ and a set $C_V$ such that
	\begin{equation}
		\label{eq:cutLabel}
		\Pr[\advers(\obfFN(\circNeigb), v_0, v, \phi(K, v_0)) \in \phi(K, C_V)] \leq \negl(\lambda).
	\end{equation}
		We define $C_V \subset V$ to be
	\begin{equation*}
		\set{ u \mid (w, u) \in C_E \text{ and } u \text { on the side of } v} \bigcup \set{ v \mid (w, u) \in C_E \text{ and } u \text { on the side of } v}.
	\end{equation*}
	In words, $C_V$ are the vertices just adjacent to the cut and on the same side as $v$.
	\begin{proof}
		We will show that if $\advers$ can break \cref{eq:cutLabel}, then we can construct an extractor,
		$E$, which finds a path from $v_0$ to $v$ with non-negligible probability.

		Assume that for every possible cut, $\advers$ is able to produce a single label in this cut for a vertex $w$.
		Then, we note that there must be at least 1 path from $v_0$ to $w$ and $v$ as otherwise, $w$ would not be in the cut.
		Moreover, we note that $\advers$ must be able to produce a label for all vertices on at least one path
		from $v_0$ to $w$ as otherwise, we can change the cut to include the edges between where
		$\advers$ is able to produce a label and not able to produce a label. Using the same argument,
		we can show that $\advers$ must be able to produce all labels on a path from $w$ to $v$.

		Note that $\advers$ is not given the specific cut $C_E$ but rather $C_E$ is chosen based off of the adversary.
		So, we can build an extractor to do the following:
		% TODO: is this selective security?????
		\begin{enumerate}
			\item Create an iO obfuscated circuit with a random key, $K'$, for $\circNeigb$ and create circuit $\obfFN(\circNeigb)$
			as well as $\phi(K', v_0)$
			\item Run $\advers(\obfFN(\circNeigb), v_0, v, \phi(K', v_0))$ to get all labels $\phi(K', v_0), \dots \phi(K', v)$
			for some path from $v_0$ to $v$.
			\item Recreate the path from $v_0$ to $v$ via checking which vertex matches to adjacent labels in the path:
			% TODO: this assumes no cycles!!!
			I.e.\ starting with $\ell = 0$, we can learn the $\ell + 1$ vertex via finding $j \in [d]$ such that
			$\circNeigb(\phi(K', v_\ell), v_\ell)_j \in \set{\phi(K', v_0), \dots, \phi(K', v)}$
			 and then setting $v_{\ell + 1} = \Gamma(v_\ell)_j$.
		\end{enumerate}
	\end{proof}
\end{lemma}

We can look at \cref{lemma:cutBaseCase} as a ``base case'' of sorts. We now inductively build up a series of games
such that $\advers$ cannot find any label in $V_1$ where $V_1$ are the vertices on side of the cut (as defined in \cref{lemma:cutBaseCase})
which contain $v$.

\begin{lemma}[Inductive Game Hypothesis]
	Let $H \subset V$ be a ``hard'' set of vertices such that $\advers$ cannot, with non-negligible probability, produce 
	$\phi(K, h)$ where $h \in H$. Note that the base case has $H = C_V$. Then, 
	for any $v \notin H$ and $w \in \Gamma(h)$ for all $h \in H$, we have that
	\begin{equation*}
		% TODO: change all to just <
		% TODO: maybe we nee to fix \eps
		\Pr[\advers(\obfFN(\circNeigb), v_0, w, \phi(K, v_0)) = \phi(K, w)] < \negl(\lambda).
	\end{equation*}
	\begin{proof}
		We are going to use a series of indistinguishable hybrids along with the circuit defined in \ref{alg:neighbHyb1} to show the above
		\begin{itemize}
			\item $\Hyb_0$: In the first hybrid, the following game is played
				\begin{enumerate}
					\item $K \gets \binSet^{\lambda'}$ and $\phi(K, v_0) = (F(K, v_0), v)$ where $K$ is some fixed secret drawn from a random distribution
					\item The challenger generates $\mathcal{O}(\circNeigb)$ and gives the program to $\advers$
					\item The challenger gives the adversary $w^*$ in plaintext.
					\item $\advers$ outputs guess $g$ and wins if $g = \phi(K, w^*)$ %hmmm... do we give w?
				\end{enumerate}
			
			\item $\Hyb_1$: We replace $\circNeigb$ with $\circNeigb^{w^*}$ as defined in \ref{alg:neighbHyb1}.
			Fix the constant $z^* = f(F(K, w^*))$
			% and $y^*_1 = f(F(K, \Gamma(w^*)_1)), \dots, y^*_d = f(F(K, \Gamma(w^*)_d))$.

			\item $\Hyb_2$: Set $z^* = f(t)$ where $t$ is chosen at random %TODO: specify field size
		\end{itemize}
		Finally, we can note that if $\Hyb_0 \compInd \Hyb_2$,
		\begin{equation*}
			\Pr[\advers(\circNeigb, v_0, w, \phi(K, v_0)) = \phi(K, w)] 
			\compInd
			\Pr[\advers(\circNeigb^*, v_0, w, \phi(K, v_0)) = \phi(K, w)]
		\end{equation*}
		where $z^*$ in $\circNeigb^*$	is the image on a OWF of a randomly chosen point.
		Thus, if $\advers$ can produce $\phi(K, v) = (\sigma_v, v)$, then $\advers$
		can find a preimage for $z*$ under $f$ and thus break the security of a one way function.

		We prove the indistinguishably of the hybrids in \cref{lemma:hybA} and \cref{lemma:hybB}.
	\end{proof}
\end{lemma}

% Hmmmmm we can still not fully use eO correctly, there may be more than one
% input which messes us up...
% I guess that we can simply **require** each vertex to have a polynomial number of inputs...
% Hmmm... also still need to see if DAGs are required!
% Maybe to remove DAG we have two functions with different proofs?: 1 on the way forward
% and 1 on the way "backwards"
% TODO: for now j assume DAG with limited (poly) number of inputs

\begin{lemma}
	\label{lemma:hybA}
	AAA
\end{lemma}

\begin{lemma}
	\label{lemma:hybB}
	AAA
\end{lemma}

\begin{algorithm}[H]
	\caption{
		Circuit for the neighbor function, $\circNeigb^{w^*, \Gamma(w^*)_1, \dots, \Gamma(w^*)_d}$ with punctured PRF key
		$K(\set{w^*})$ and constant $z^*$%, y^*_1, y^*_2, \dots, y^*_d$
	}
	\begin{algorithmic}[1]
		\Function{$\circNeigb$}{$f(\sigma_v), v$}
			\If{$v \neq w$ and $f(\sigma_v) \neq f(F(K, v))$}
				\State \Return $\bot$
			\EndIf
			\If{$v = w$ and $f(\sigma_v) \neq z^*$}
				\State \Return $\bot$ 
			\EndIf
		 	\If{$\Gamma(v) = \emptyset$}
				\State \Return $\bot$
			\EndIf
			% \If{$v = w$}
			% 	\State \Return $z^*_1, z^*_2, \dots, z^*_d$
			% \EndIf
			\State $u_1, \dots u_d = \Gamma(v)$
			\If{For some $j \in [d]$, $u_j = w^*$}
				% Hmmm cannot *actually* return the punctured point
				% Maybe need to use weak eO here...
			 	\State Set $F(K, u_j) = \bot$
			\EndIf
			\State \Return $F(K, u_1), F(K, u_2), \dots, F(K, u_d)$
		\EndFunction
	\end{algorithmic}
	\label{alg:neighbHyb1}
\end{algorithm}

 

	\begin{lemma}
		The game in $\Hyb_1 (1a)$ is indistinguishable from $\Hyb_0$.
		\begin{proof}
			As the functionality of $\circNeigb$ in $\Hyb_0$ equals that of $\Hyb_1 (1a)$,
			we have indistinguishable simply from the definition of indistinguishable obfuscation.
		\end{proof}
	\end{lemma}

	\begin{lemma}
		The game in $\Hyb_1 (1b)$ is indistinguishable from $\Hyb_1 (1a)$.
		\begin{proof}
			Here we argue that if the game in $\Hyb_1 (1b)$ is distinguishable from
			$\Hyb_1 (1a)$, then we can construct an adversary, $\adversB$, which can break the security of the PRF
			at the punctured point. 

			% TODO: wait is this trivial??? Like just use the extractor definition???
			% But then wait, something is not right here
			% Something 

			% TODO: change game
		\end{proof}
	\end{lemma}

	\begin{lemma}
		The game in $\Hyb_1 (2a)$ is indistinguishable from $\Hyb_0$ and, by the inductive hypothesis, all previous hybrids.
		\begin{proof}
			Again, we have that the circuit for $\circNeigb$ is the same in $\Hyb_0$ and $\Hyb_1 (2a)$.
			Thus, by the definition of indistinguishable obfuscation, these games are indistinguishable.
		\end{proof}
	\end{lemma}

	\begin{lemma}
		The game in $\Hyb_1 (2b)$ is indistinguishable from $\Hyb_1 (2a)$ and, by the inductive hypothesis, all previous hybrids.
		\begin{proof}
			TODO: PRF security + extractor part
		\end{proof}
	\end{lemma}

	