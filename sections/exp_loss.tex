\section{A sketch for the boys}
\newcommand{\labelFunc}{\phi}
\newcommand{\imageFn}{\text{Image}}
\newcommand{\pathSuffix}{\text{Suff}}

\subsection{DAG Randomized Traversal}
Say that we have a sparse, potentially exponentially sized, graph $\graph = (\verts, \edges)$
and $\forall v \in \verts, \deg(v) = d$.
We also require that $\graph$ is equipped with a neighbor function, $\Gamma$, which can be computed in polynomial time.
% (TODO: padding).
We define a (pseudo) randomized and keyed labelling function $\phi: \verts \times \binSet^\lambda \rightarrow \binSet^{\poly(\lambda)}$ 
such that given, $\phi(K, v_0)$ for root $v_0$, an adversary, $\advers$, which does not know a path from $v_0$ to $v$,
\begin{equation}
	\label{eq:guessPhi}
	\Pr[\advers(\circNeigb, v_0, v, \labelFunc(K, v_0)) \in \imageFn(\labelFunc(K, v))] \leq O(v)\eps
\end{equation}
for some fixed $\eps \leq \negl(\lambda)$ and function $\circNeigb$ where $\circNeigb(\labelFunc(K, u)) = \labelFunc(K, \Gamma(u)_1), \dots, \labelFunc(K, \Gamma(u)_d)$
if $\Gamma(u) \neq \emptyset$ and otherwise $\Gamma(u)$ returns a $0$ string of length $d |\labelFunc(K, \cdot)|$.

\subsection{Instantiation}
We define $\labelFunc(K, v)$ to be as follows:
\begin{enumerate}
	\item Let $r_1, r_2 \randomGet \binSet^\lambda$ or $r_1, r_2$ is drawn from a pseudorandom distribution. % TODO: hmm
	\item Return $\FEEnc(\MPK, (K, v, r_2))$ where encryption is done with randomness from $r_1$.
\end{enumerate}

We can now define, $\circNeigb$.
\begin{algorithm}[H]
	\caption{
		The circuit for the neighbor function, $\circNeigb$.
	}
	\begin{algorithmic}[1]
		\Function{$\inner_i$}{$K, v, r$}
		 	\If{$\Gamma(v) = \emptyset$}
				\State \Return $0 \in \binSet^*$ %TODO: define star to just be the same length....
			\EndIf
			\State $u_1, \dots, u_d = \Gamma(v)$
			\State $u = u_i$
			\State $r_1, r_2 = \PRG(r)$
			\label{alg:neighb:prg}
			\State \Return $\FEEnc(\MPK, (u, K, r_2))$ where we encrypt with randomness from $r_1$.
		\EndFunction
		\Function{$\circNeigb$}{$\phi(K, v)$}
		 	\For{$i \in [d]$}
				\State $u_i = \Dec(\SK_{\texttt{inner}_i}, \phi(K, v))$
			\EndFor
			\State \Return $(u_1, \dots, u_d)$
		\EndFunction
	\end{algorithmic}
	\label{alg:neighb}
\end{algorithm}

% Before showing that our definition of $\labelFunc$ and $\circNeigb$ satisfy \cref{eq:guessPhi}, we first must show that an adversary cannot find $K$.
% \begin{lemma}
% 	\label{lem:findK}
% 	Let $\advers$ be a PPT adversary which can find $K$ with probability $N\eps$.
% 	Then, there exists a PPT adversary, $\adversB$ which can break the FE scheme with probability $\eps$.	
% 	Or, given that the FE scheme is $\eps$ secure, then,
% 	\begin{equation*}
% 		\Pr[\advers(\Gamma, \circNeigb, v_0, \labelFunc(K, v_0)) = K] \leq O(|\verts|)\eps
% 	\end{equation*}
% 	\begin{proof}
% 		We first prove the above but in the case of selective security. I.e.\ the adversary has to fix
% 		its query path at the start.
% 		We then use standard complexity leveraging techniques to achieve adaptive security.

% 		We proceed via a series of hybrids. Note that for $|\verts| \geq \exp(\lambda)$, we require exponential hardness for the FE scheme.
% 		As such, let modified security parameter $\lambda' = \lambda + O(\log |\verts|)$.
% 		\begin{itemize}
% 			\item $\Hyb_0$: In the first hybrid, the following game is played
% 				\begin{enumerate}
% 					\item $K \randomGet \binSet^{\lambda'}$ and $\MPK, \SK \gets \FESetup(1^{\lambda'})$.
% 					\item The challenger generates $\SK_{\inner_i} \gets \FEKeygen(\MSK, \inner_i)$ for $i \in [d]$ and gives these keys to $\advers$
% 					\item The challenger picks random $r_1, r_2 \randomGet \binSet^{\lambda'}$ and generates $\phi(K, v_0) = \FEEnc(\MPK, (K, v_0, r_2))$ using $r_1$ as the random coins and gives $\phi(K, v_0)$ to $\advers$.
% 					\item $\advers$ outputs guess $K'$ and wins if $K = K'$
% 				\end{enumerate}
% 			\item $\Hyb_1$: We replace the \emph{entire graph} with its simulated counterpart,
% 			starting at the root and moving to the leaves.
% 			TODO: we basically replace the ct with simulated, replace the output with true randomness,
% 			giving us a valid ct\_next, then we move on to the next ct, etc.

% 			\item $\Hyb_1$: We replace the \emph{entire graph} with its simulated counterpart.
% 			We start with the leaves and then ``work backwards'' doing two steps at a time.
% 			Let $U_1, \dots, U_D$ be a partition of $V$ such that for all $u \in U_i$, the minimum distance from $u$ to a the root is $i$.
% 			Starting with $j = D$, we do the following
% 			\begin{enumerate}
% 				\item For vertex $u \in U_j$, we invoke the FE simulator to get
% 				\begin{align*}
% 						\FECT_{u}' \gets \Sim\big(1^\lambda, &1^{TODO}, \MPK, (\inner_1, \dots, \inner_d), (\SK_{\inner_1}, \dots \SK_{\inner_d}),\\
% 							&(\inner'_1(u), \dots, \inner'_d(u))\big)
% 				\end{align*}
% 				where $\inner'_i$ is the same as $\inner_i$ except that instead of using $r_1$ as randomness for $\FEEnc$, we use $r_1'$ which is true randomness that is fixed for $u$.
% 				Moreover, $\inner'_i$ uses true, fixed randomness, $r_2'$, rather than $r_2$.
% 				\item For all $w \in \verts$ where $\Gamma(w)_i = u$, we replace $\inner'_i(\phi(K, w))$ with $\FECT_{u}'$.
% 				\label{alg:proof:neighb:replace}
% 				\item If $j \neq 1$, we go back to step 1 with $j = j - 1$.
% 			\end{enumerate}
% 		\end{itemize}
% 		Assuming $\Hyb_1$ and $\Hyb_0$ computational indistinguishable, if we can guess $K$, then we can build an adversary $\adversB$ which can show that $\Hyb_0$ and $\Hyb_1$ are not indistinguishable.
% 		An adversary $\adversB$ can simply feed in the observables of $\Hyb_0$ into $\advers$ and $\Hyb_1$ into $\advers$.
% 		As $\Hyb_1$ is independent of K, we necessarily have that $\Pr[\advers(\Hyb_1) = K] \leq \negl(\lambda)$.
% 		Then, we can note that if $\advers$ can guess $K$ then, $\Pr[\advers(\Hyb_0) = K] > \negl(\lambda)$.
% 		Thus, we have that $\Pr[\advers(\Hyb_0) = K] - \Pr[\advers(\Hyb_1) = K] > \negl(\lambda)$ and thus $\Hyb_0$ and $\Hyb_1$ are not computationally indistinguishable.	\\
% 		% TODO: do we need a CRHF of K to check whether or not we are in \Hyb_0 or \Hyb_1?


% 		Now we just have to show that $\Hyb_0 \compInd \Hyb_1$. First, we can note
% 		that $\inner_i(\phi(K, u)) \compInd \inner_i'(\phi(K, u))$ as if these distributions are indistinguishable,
% 		we can break the security of the $\PRG$ used in line \ref{alg:neighb:prg} of \cref{alg:neighb}. 
% 		% TODO: more concrete?
% 		Now, we can note that if $\FECT_u = \phi(K, u)$, we have that $\FECT_u \compInd \FECT_u'$ are indistinguishable
% 		if $\inner_i(\FECT_u) \compInd \inner_i(\FECT_u')$ for all $i \in [d]$. We can note that this holds true for
% 		$u$ if $u$ is a leaf as $\inner_i(u) = 0$ and thus we can invoke the security of the FE simulator.
% 		Then, by step \ref{alg:proof:neighb:replace} in the above hybrid, we inductively create the hybrid
% 		such that $\inner_i(\FECT_w) \compInd \inner_i(\FECT_w')$. Thus, as we work backwards from the leaves,
% 		we have that $\FECT_u \compInd \FECT_u'$ for all $u \in \verts$. Assuming that the FE scheme is $\eps$ secure
% 		and the distance from the distributions of $\inner_i(\FECT_u)$ and $\inner_i'(\FECT_u)$ is at most $\eps$,
% 		then by the triangle inequality, we have that the distance between $\Hyb_0$ and $\Hyb_1$ is at most
% 		$O(|\verts|)\eps$.
% 		More formally, we have that for any PPT adversary, $\adversB$,
% 		\begin{equation*}
% 			\big| \Pr[\adversB(\Hyb_0) = 1] - \Pr[\adversB(\Hyb_1) = 1] \big| \leq O(|\verts|)\eps.
% 		\end{equation*}
% 	\end{proof}
% \end{lemma}

% \begin{lemma}
% 	\label{lem:withoutCirc}
% 	Let $\advers$ be a PPT adversary which can find $K$ with probability $N\eps$.
% 	Then, there exists a PPT adversary, $\adversB$ which can break the FE scheme with probability $\eps$.	
% \end{lemma}


Now, we will prove \cref{eq:guessPhi}. For convenience, we will restate \cref{eq:guessPhi}:
\begin{equation*}
	\Pr[\advers(\circNeigb, v_0, v, \labelFunc(K, v_0)) \in \imageFn(\labelFunc(K, v))] \leq O(v)\eps
\end{equation*}
for all PPT $\advers$.

\begin{proof}[Proof of \cref{eq:guessPhi}]
	Let $P$ be the set of all paths from $v_0$ to $v$. Then, as the adversary does not know any path from $v_0$ to $v$,
	we have that for each $p \in P$ where $p = v_0, \dots, v$, there is some suffix of $p$, $p'$, which the adversary does not know.

	We are going to use a similar proof technique as in \cref{lem:findK} where we use a hybrid argument.
	\begin{itemize}
		\item $\Hyb_0$: In the first hybrid, the following game is played
			\begin{enumerate}
				\item $K \randomGet \binSet^{\lambda'}$ and $\MPK, \SK \gets \FESetup(1^{\lambda'})$.
				\item The challenger generates $\SK_{\inner_i} \gets \FEKeygen(\MSK, \inner_i)$ for $i \in [d]$ and gives these keys to $\advers$
				\item The challenger chooses a $v$ and gives the adversary $v$ in plaintext.
				\item The challenger picks random $r_1, r_2 \randomGet \binSet^{\lambda'}$ and generates $\phi(K, v_0) = \FEEnc(\MPK, (K, v_0, r_2))$ using $r_1$ as the random coins and gives $\phi(K, v_0)$ to $\advers$.
				\item $\advers$ outputs guess $g$ and wins if $g \in \phi(K, v)$
			\end{enumerate}
		\item $\Hyb_1$: We replace $\phi(K, v_0)$ with a simulated cipher-text using the simulator $\Sim_2$ 
		$\MPK$ with its simulated counterpart using $\Sim_0$, and $\SK_{\inner_i}$ with its simulated counterpart using $\Sim_3$ as defined in \cref{eq:FEsec}.

		\item $\Hyb_{2a}$: For any input into $\Sim_2$ via $\Pi^{K, w, r}$ where $w \in \Vert$
		and $r$ is random, we replace the output of $\inner_i$ with $\inner_i'$ which uses true randomness $r_1^*, r_2^*$ in stead of $r_1, r_2$.
		For any call to $\circNeigb(\phi(K, w))$ by $\advers$ for $w \in \verts$,
		we replace the output of $\inner_i$ with $\inner_i'$ which uses true randomness $r_1^*, r_2^*$ in stead of $r_1, r_2$.
		This is equivalent to changing $\Pi^m$ to ${\Pi^m}'$ in \cref{eq:FEsec} where ${\Pi^{m}}'$ is 
		the list $\big(\inner_1, \inner_1'(\cdot), \dots, \inner_d, \inner_d'(\cdot)\big)$.
		Note that this gives us that $\inner_i'(K, w, r) = \phi(K, u) = \FEEnc(\MPK, (K, u, r^*_2))$ where $u = \Gamma(w)_i$.

		\item $\Hyb_{2b}$: For any call by $\advers$ to $\inner_i'(K, w, r) = \FECT$, we replace $\FECT$ with
		with $\FECT'$ where $\FECT'$ is the output of $\Sim_2$ with input ${\Pi^{(K, u, r)}}'$ where $u = \Gamma(w)_i$.

		Note that the replacement of $\Hyb_{2a}$ and $\Hyb_{2b}$ are repeated multiple times.
		Specifically, these replacements are repeated at most $\alpha$ times where $\alpha$ is the number
		of unique times $\advers$ runs $\FEDec(\SK_{\inner_i}, \phi(K, w))$.


		% Further, we use the simulator $\Sim_2$ to replace $\phi(K, u)$ with its simulated counterpart $\Sim_2(\mathbf{st}_1, \Pi^{(K, u, r^*_2)})$ with its simulated counterpart.
		% TODO: hmmm... is this the exponential part?????

		\item $\Hyb_3$: Let $\mathcal{P}$ be the set of all paths from $v_0$ to $v$. For each path $P \in \mathcal{P}$
		where $P$ is an ordered list of connected vertices, we have that the adversary does not know
		some part of $P$. We can note that this implies that $\advers$ never queries
		$\inner_i(w^u)$ where $u = \Gamma(w^u)_i$ for some $u \in P$ and the adversary knows a path from $v_0$ to $w$.
		We can see this because if there is no $u \in P$ such that $\advers$ never queries $\inner_i(w^u)$,
		then the adversary knows a path from $v_0$ to $v$. Define $\pathSuffix(P)$ to be the path 
		which starts at $u$, ends at $v$, and is a suffix of $P$.
		We now inductively build up a series of hybrids to show that a hybrid distribution
		which ``erases'' $\phi(K, v)$ from $\inner_i$ is indistinguishable from the above hybrid.
		\begin{itemize}
			\item For the base case, let $U = \set{u_1, \dots, u_{\|\mathcal{P}\|} }$
			where $u$ is the first vertex in $P$ such that $\advers$ never queries $\inner_i(w^u)$ as defined above.
			Then, replace $\inner_i'(\cdot)$ with $\inner_i^*(\cdot)$ in $\Pi_m$ such that $\inner_i^*(w) = \inner_i'(w)$
			if $w \neq w^u$ for $u \in U$ and otherwise $\inner_i^*(w^u) = \bot$. We can note that this hybrid is indistinguishable as
			$\inner_i'$ only changes for input ciphertexts which the adversary never queries.
			% TODO: need to show that the adversary does not know \phi(K, w^u)' where this is a different encryption
			% of \phi(K, w^u) than one seen in a path
			\item For the $\ell$-th inductive step, we are going to assume that we are given a hybrid such that $\inner_i^\ell$ such that $\inner_i^\ell(w^u) = \bot$
			for $u \in U^\ell$ where $U^\ell$ where $U^\ell = \bigcup_{P \in \mathcal{P}} \pathSuffix(P)_1, \dots, \pathSuffix(P)_\ell$
			and otherwise $\inner_i^\ell(\cdot) = \inner_i'(\cdot)$. We now show that if $\advers$ can distinguish
			between a hybrid with $\inner_i^\ell(\cdot)$ and $\inner_i^{\ell + 1}(\cdot)$, then the adversary can break
			the non-malleability of the FE scheme.
			We defer this proof to \cref{lem:NMIndep}.
			% TODO::!!
		\end{itemize}
	\end{itemize}
	Finally, we can note that by the indistinguishably of $\Hyb_0$ and $\Hyb_5$,
	\begin{equation*}
		\Pr[\advers(\circNeigb, v_0, v, \phi(K, v_0)) \in \imageFn(\phi(K, v))] 
		\compInd
		\Pr[\advers(\circNeigb', v_0, v, \phi(K, v_0)) \in \imageFn(\phi(K, v))] 
	\end{equation*}
	where $\circNeigb'$ is $\circNeigb$ except that $\circNeigb'$ uses $\inner_i^p$ where $p = \max_{P \in \mathcal{P}} |P|$.
	We can note that $\circNeigb'$ returns $\bot$ for any query on $\phi(K, w^v)$ where $w^v \in \Gamma^{-1}(v)$.
	Using \cref{lem:NMIndepA}, we have that
	$$
		\Pr[\advers(\circNeigb', v_0, v, \phi(K, v_0)) \in \imageFn(\phi(K, v))] \leq \negl(\lambda).
	$$
	
\end{proof}

\begin{lemma}
	\label{lem:hybrOneToFour}
	$\Hyb_0 \compInd \Hyb_{2b}$.	
	\begin{proof}
		First we show that $\Hyb_0 \compInd \Hyb_1$. Note that if $\advers$ can distinguish
		between $\Hyb_0$ and $\Hyb_1$ then an adversary can distinguish between an FE scheme and its simulated counterpart
		where $m$ is fixed to $(K, v_0, r)$. We can see this as $\Hyb_1$ is  direct simulation of the FE scheme.

		Then, if $\advers$ can distinguish $\Hyb_1$ and $\Hyb_{2a}$, then we can break the security of the $\PRG$ used in line \ref{alg:neighb:prg} of \cref{alg:neighb}.
		We can create an adversary $\adversB$ which, for some fixed $K$, distinguishes between $\FEEnc(\MPK, (K, u r_2))$ with random coins $r_1$ where $r_1, r_2 = \PRG(r)$
		and $\FEEnc(\MPK, (K, u, r_1^*))$ encrypted with random coins $r_2^*$ where $r_1^*, r_2^*$ are truly random.

		Then, if $\advers$ can distinguish any transformation from $\Hyb_{2a}$ to $\Hyb_{2b}$, then we can break the security of the FE scheme.
		We can see this by noting that if we fix $m = (K, w, r)$ for random $r$ and $K$,
		then $\advers^{\Sim_3^{U_m(\cdot)}}(\FECT)$ is distinguishable
		and $\advers^{\sim_3^{u_m(\cdot)}}(\FECT')$ where $\FECT$ is the real cipher-text and $\FECT'$ is simulated.
		We can then note that if the above are distinguishable, then
		$\advers^{\text{KeyGen}(\MSK, \set{\inner_1, \dots \inner_d})}(\FECT)$ 
		and $\text{KeyGen}(\MSK, \set{\inner_1, \dots \inner_d})$ are distinguishable
		as $\advers^{\text{KeyGen}(\MSK, \set{\inner_1, \dots \inner_d})}$ can simply simulate
		$\advers^{\Sim_3^{U_m(\cdot)}}(\FECT)$.

		Then, if $\advers$ can distinguish any transformation from $\Hyb_{2b}$ to $\Hyb_{2a}$, then we can break the security of a PRG
		in the same manner as distinguishing $\Hyb_1$ and $\Hyb_{2a}$.

		By the chain rule, we get that $\Hyb_0$ and $\Hyb_{2b}$ are indistinguishable even after a repeated
		number of sequential invocations of the transformation in $\Hyb_{2a}$ and $\Hyb_{2b}$.
	\end{proof}
\end{lemma}


\begin{lemma}
		\label{lem:NMIndep}
		Let $\advers$ be a PPT adversary and assume that we have a non-malleable and simulation secure
		FE scheme. Then, we have that the inductive step of $\Hyb_3$ holds.
		\begin{proof}
			We construct an adversary $\adversB$ that can break NM security using $\advers$ if $\advers$ can distinguish between the hybrids
			in the inductive step.
			Note that in order to distinguish between the hybrids, $\advers$ must have queried $\inner^\ell_i$ or $\inner^\ell_{i + 1}$ on $\phi(K, w^u)$ where $u \in \set{\pathSuffix(P)_{\ell + 1} \mid P \in \mathcal{P}}$
			as the this is the only difference between the hybrids.
			Thus, we see that $\advers$ is able to produce to produce $\FECT \in \phi(K, w^u)$.
			By definition of $\inner_i^\ell$ though, we know that $\inner_i^\ell(\phi(k, q)) \neq \phi(K, w^u)$
			for any $q \in \verts$ as we define $\inner_i^\ell(K, q) = \bot$ if $\inner_i'(K, q) = \phi(K, w^u)$.
			Thus, the adversary has to be able to produce $\FECT \in \phi(K, w^u)$ without calling $\circNeigb$.

			Thus, if $\advers(w^u, v_0, \circNeigb, \phi(K, v_0))$ can produce $\FECT \in \phi(K, w^u)$,
			we can have $\adversB(\phi(K, v_0), \phi(K, q_1), \dots,$ $ \phi(K,q_{\poly(\lambda)}))$ 
			produce $\phi(K, w^u)$ where $q_1, \dots, q_{\poly(\lambda)}$ are all the vertices that $\advers$ has queried $\circNeigb$ on.
			$\adversB$ simply has to invoke $\Sim_1$ to create a simulated set of function keys for $\inner_i$ for all $i \in [d]$
			and can then simulate $\circNeigb$.

			We can then have $\adversB$ invoke $\Sim_1$ to create a simulated function key for $\SK_{\inner_i}'$
			and thus a simulated $\circNeigb'$.
			$\adversB$ then gives $\advers$ $(w^u, v_0, \circNeigb', \phi(K, v_0))$.
			we can then break \cref{eq:NMRel} as $\advers$ is able to create an encryption of $\phi(K, w^u)$ with non-negligible probability while the simulator cannot.
		\end{proof}
\end{lemma}

\begin{lemma}
		\label{lem:NMIndepA}
		Define $\circNeigb'$ where $\circNeigb'$ is defined as in \cref{alg:neighb} except that
		for some set $U \subset \verts$, $\circNeigb(w^u)_i = \bot$ for all $w^u \in \verts$ such that $u = \Gamma(w^u)_i$ for some $u \in U$.
		In words, the parent of all $u \in U$ do not return $\phi(K, u)$ when queried on $\circNeigb'$.
		Then, assuming the non-malleability and simulation security of FE, we have that for all PPT $\advers$
		and all $u \in U$,
		\begin{equation}
			\Pr[\advers(\circNeigb', v_0, u, U, \phi(K, v_0)) \in \imageFn(\phi(K, u))] \leq \negl(\lambda).
		\end{equation}
		\begin{proof}
			We construct an adversary $\adversB$ that can break NM security using $\advers$ if $\advers$ can
			produce $\FECT \in \phi(K, u)$ for some $u \in U$.
			
			If $\advers(w^u, v_0, \circNeigb, u, U, \phi(K, v_0))$ can produce $\FECT \in \phi(K, u)$,
			we can have $\adversB(\phi(K, v_0), \phi(K, q_1), \dots,$ $ \phi(K,q_{\poly(\lambda)}))$ 
			produce $\phi(K, u)$ where $q_1, \dots, q_{\poly(\lambda)}$ are all the vertices that $\advers$ has queried $\circNeigb'$ on.
			$\adversB$ simply has to invoke $\Sim_1$ to create a simulated set of function keys for $\inner_i$ for all $i \in [d]$
			and can then simulate $\circNeigb'$ with these function keys.

			We can then have $\adversB$ invoke $\Sim_1$ to create a simulated function key for $\SK_{\inner_i}'$
			and thus a simulated $\circNeigb^{*}$.
			$\adversB$ then gives $\advers$ $(w^u, v_0, \circNeigb^{*}, \phi(K, v_0))$.
			We can then break \cref{eq:NMRel} (this is supposed to be the NM relationship equation) as $\advers$ is able to create an encryption of $\phi(K, w^u)$ with non-negligible probability while the simulator cannot.
		\end{proof}
\end{lemma}
