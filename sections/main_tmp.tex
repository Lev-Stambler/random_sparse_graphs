\section{DAG Label Obfuscation from Additive Overhead iO}
\subsection{DAG Randomized Traversal}
Say that we have a sparse, potentially exponentially sized, graph $\graph = (\verts, \edges)$ with polynomial depth $D$,
and forall $v \in \verts, \deg(v) \leq d$. Moreover, for simplicity,
assume that for all $v$, 
$$
\deg^{-1}(v) = \big|\set{u \in V \mid \exists j \in [d], \Gamma(u)_j = v}\big| \leq d.
$$
In words, there are at most $d$ edges into a vertex. As a note, our construction just requires
that $\deg^{-1}(\cdot) = O(1)$ but for the sake of simplicity we fix $\deg^{-1}(\cdot) \leq d$.

We also require that $\graph$ is equipped with a neighbor function, $\Gamma$, which can be computed in polynomial time.
% (TODO: padding).
We define a randomized and keyed labelling function $\phi: \binSet^\lambda \times \verts \rightarrow \binSet^{\poly(\lambda)}$ 
such that given, $\phi(K, v_0)$ for root $v_0$, a PPT adversary which runs in time at most $T(\lambda)$, $\advers$, which does not know a path from $v_0$ to $v$,
% TODO: defn extractor??
\begin{equation}
	\label{eq:guessPhi}
	\Pr[\advers(\mathcal{O}(\circNeigb), v_0, v, \labelFunc(K, v_0)) = \labelFunc(K, v)] \leq \epsilon
\end{equation}
for function $\circNeigb$ where $\circNeigb(\labelFunc(K, u)) = \labelFunc(K, \Gamma(u)_1), \dots, \labelFunc(K, \Gamma(u)_d)$ and the circuit is padded to size $\circSize$.
if $\Gamma(u) \neq \emptyset$ and otherwise $\Gamma(u)$ returns a $\bot$ string.
We fix the adversary's advantage to $\epsilon < \poly(\lambda)$ and runtime to $T(\lambda) \leq \poly(\lambda, \frac{1}{\eps})$
as we will need to show
that a set of a potentially exponential number of games \emph{does not have exponential security loss}
nor or \emph{reduce down to security against an exponentially strong adversary}.

\subsection{Instantiation}
We define 
$
	\labelFunc(K, v) = F(K, v)
$ for $K \randomGet \set{0, 1}^\lambda$, and we can now define $\circNeigb$:
\begin{algorithm}[H]
	\caption{
		The circuit for the neighbor function, $\circNeigb$ padded out to size $\circSize$.
	}
	\begin{algorithmic}[1]
		\Function{$\circNeigb$}{$X, v$}
			\If{$f(X) \neq f(F(K, v))$}
				\State \Return $\bot$
			\EndIf
		 	\If{$\Gamma(v) = \emptyset$}
				\State \Return $\bot$
			\EndIf
			\State $u_1, \dots u_d = \Gamma(v)$
			\State \Return $F(K, u_1), F(K, u_2), \dots, F(K, u_d)$
		\EndFunction
	\end{algorithmic}
	\label{alg:neighb}
\end{algorithm}

% TODO: more concrete somewhere with overhead
We are going to show that \cref{eq:guessPhi} for $\circSize = O(D \cdot \text{overhead})$ where ``overhead'' is the additive overhead of $\iO$ obfuscation.
We will do this by first showing that the non-existence of an extractor to find a path from $v_0$ to $v$ implies that $\advers$
necessarily does not know $\phi(K, c)$ for a $c \in C_V \subset V$ where the vertices in $C_V$ border 
a graph cut which separates $v_0$ and $v$. Note that the base case holds for all $S \geq \poly(\lambda)$. 


Then, we inductively build up a series of games to show that
$\advers$ cannot learn \emph{any} $\phi(K, v)$ for $v \in V_1$ where $V_1$ are the vertices on the side of the cut containing $v$.
At each inductive step, we restrict the security game to hold for $S \geq O(\inductiveInd \cdot \text{overhead})$ where $\inductiveInd$ is the number of calls to induction.
% TODO: note no more than D inductive calls

% TODO: define P way above
\begin{lemma}[Base Case Game]
	\label{lemma:cutBaseCase}
	Assuming that there is no extractor $E$ such that $\Pr[E(\Gamma, v_0, v) = P] \geq \frac{1}{p(\lambda)}$
	where $P \in \pathSet$, then for any PPT $\advers$, there exists some graph cut 
	$C_E \subset E$ which separates $v_0$ and $v$ and a set $C_V$ such that
	\begin{equation}
		\label{eq:cutLabel}
		\Pr[\advers(\obfFN(\circNeigb), v_0, v, \phi(K, v_0)) \in \phi(K, C_V)] < \eps.
	\end{equation}
		We define $C_V \subset V$ to be
	\begin{equation*}
		\set{ u \mid (w, u) \in C_E \text{ and } u \text { on the side of } v} \bigcup \set{ w \mid (w, u) \in C_E \text{ and } w \text { on the side of } v}.
	\end{equation*}
	In words, $C_V$ are the vertices just adjacent to the cut and on the same side as $v$.
	\begin{proof}
		We will show that if $\advers$ can break \cref{eq:cutLabel}, then we can construct an extractor,
		$E$, which finds a path from $v_0$ to $v$ with non-negligible probability.

		Assume that for every possible cut, $\advers$ is able to produce a single label in this cut for a vertex $w$.
		Then, we note that there must be at least 1 path from $v_0$ to $w$ and from $w$ to $v$ as otherwise, $w$ would not be in the cut.
		Moreover, we note that $\advers$ must be able to produce a label for all vertices on at least one path
		from $v_0$ to $w$ as otherwise, we can change the cut to include the edges between where
		$\advers$ is able to produce a label and not able to produce a label. Using the same argument,
		we can show that $\advers$ must be able to produce all labels on a path from $w$ to $v$.

		Note that $\advers$ is not given the specific cut $C_E$ but rather $C_E$ is chosen based off of the adversary.
		So, we can build an extractor to do the following:
		% TODO: is this selective security?????
		\begin{enumerate}
			\item Create an iO obfuscated circuit with a random key, $K'$, for $\circNeigb$ and create circuit $\obfFN(\circNeigb)$
			as well as $\phi(K', v_0)$
			\item Run $\advers(\obfFN(\circNeigb), v_0, v, \phi(K', v_0))$ to get all labels $\phi(K', v_0), \dots \phi(K', v)$
			for some path from $v_0$ to $v$.
			\item Recreate the path from $v_0$ to $v$ via checking which vertex matches to adjacent labels in the path:
			% TODO: this assumes no cycles!!!
			I.e.\ starting with $\ell = 0$, we can learn the $\ell + 1$ vertex via finding $j \in [d]$ such that
			$\circNeigb(\phi(K', v_\ell), v_\ell)_j \in \set{\phi(K', v_0), \dots, \phi(K', v)}$
			 and then setting $v_{\ell + 1} = \Gamma(v_\ell)_j$.
		\end{enumerate}
	\end{proof}
\end{lemma}

We can look at \cref{lemma:cutBaseCase} as a ``base case'' of sorts. We now inductively build up a series of games
such that $\advers$ cannot find any label in $V_1$ where $V_1$ are the vertices on side of the cut (as defined in \cref{lemma:cutBaseCase})
which contain $v$.

\begin{lemma}[Inductive Game Hypothesis]
	Let $H \subset V$ be a ``hard'' set of vertices such that $\advers$ cannot, with non-negligible probability, produce 
	$\phi(K, h)$ where $h \in H$. Note that the base case has $H = C_V$. 
	Assuming adaptive security of constrained PRFs, one way functions, and the existence of indistinguishable obfuscation,
	we then have
	for any $w \in \Gamma(h)$ for all $h \in H$, 
	\begin{equation*}
		% TODO: change all to just <
		% TODO: maybe we nee to fix \eps
		\Pr[\advers(\obfFN(\circNeigb), v_0, w, \phi(K, v_0)) = \phi(K, w)] < \eps.
	\end{equation*}
	\begin{proof}
		We are going to use a series of indistinguishable hybrids along with the circuit defined in \ref{alg:neighbHyb1} to show the above
		\begin{itemize}
			\item $\Hyb_0$: In the first hybrid, the following game is played
				\begin{enumerate}
					\item The challenger gives the adversary $w^*$ in plaintext.
					\item $K \gets \binSet^{\lambda'}$ and $\phi(K, v_0) = (F(K, v_0), v_0)$ where $K$ is some fixed secret drawn from a uniform distribution
					\item The challenger generates $\mathcal{O}(\circNeigb)$ and gives the program to $\advers$
					\item $\advers$ outputs guess $g$ and wins if $g = \phi(K, w^*)$ %hmmm... do we give w?
				\end{enumerate}
			
			\item $\Hyb_1$: We replace $\circNeigb$ with $\circNeigb$ as defined in circuit~\ref{alg:neighbHyb1}.
			Fix the constant $z^* = f(F(K, w^*))$
			% and $y^*_1 = f(F(K, \Gamma(w^*)_1)), \dots, y^*_d = f(F(K, \Gamma(w^*)_d))$.
			\item $\Hyb_{2, 1}$
			We replace circuit~\ref{alg:neighbHyb1} with circuit~\ref{alg:neighbHyb2} where we 
			set $Y^* = (1, y)$ such that $\Gamma(y)_1 = w^*$. So then, we have that
			have $F(K, \Gamma(y)_1) = \bot$. Moreover, we set the punctured set, $S$ to $\emptyset$ (i.e.\ we do not puncture the PRF).
			\item $\Hyb_{2, j}$ for $j \in \set{2, \dots, \deg^{-1}(w^*)}$
			We replace $Y^*$ with $Y^* \cup (j, y)$ such that $\Gamma(y)_j = w^*$.
			Note after the last of these hybrids, we have that $F(K, w^*)$ is always set to $\bot$.
			\item $\Hyb_3$: We puncture the PRF at $w^*$ and set $S = \set{w^*}$.
			\item $\Hyb_4$: Set $z^* = f(t)$ where $t$ is chosen at random %TODO: specify field size
		\end{itemize}
		Finally, we can note that if $\Hyb_0 \compInd \Hyb_4$,
		\begin{equation*}
			\Pr[\advers(\circNeigb, v_0, w, \phi(K, v_0)) = \phi(K, w)] 
			\compInd
			\Pr[\advers(\circNeigb^*, v_0, w, \phi(K, v_0)) = \phi(K, w)]
		\end{equation*}
		where $z^*$ in $\circNeigb^*$	is the image on a OWF of a randomly chosen point.
		As we will show in \cref{lemma:hybA}, \cref{lemma:hybB}, and \cref{lemma:hybC},
		an adversaries advantage between games in $\Hyb_0$ and $\Hyb_3$ is at most $\epsilon / 2$.
		Thus, if $\advers$ can produce $\phi(K, v) = (\sigma_v, v)$ with advantage $\epsilon / 2$
		in $\Hyb_3$, then $\advers$
		can find a pre-image for $z^*$ under $f$ with non-negligible probability and thus break the security of a one way function.
		We then have that the advantage of the adversary in $\Hyb_0$ cannot be more than $\epsilon$.
	\end{proof}
\end{lemma}

\begin{lemma}
	\label{lemma:hybA}
	% TODO: workout
	$\Hyb_0$ and $\Hyb_1$ are distinguishable with advantage at most $\epsilon / 10$.
	\begin{proof}
		Assume towards contradiction that $\eps \in \poly(1/\lambda)$.
		Note that for all inputs $(z, v)$ to $\circNeigb$ as defined in circuit~\ref{alg:neighb} and circuit~\ref{alg:neighbHyb1}
		are equivalent and thus indistinguishable by the definition of indistinguishable obfuscation.
		So, if $\eps \in \poly(\lambda)$, then an adversary cannot distinguish the hybrids with probability more than $\epsilon / 10$.
	\end{proof}
\end{lemma}

\begin{lemma}
	\label{lemma:hybB}
	Each hybrid from $\Hyb_1$ to $\Hyb_{2, 1}$ and $\Hyb_{2, j - 1}$ to $\Hyb_{2, j}$ for $j \in 2, \dots, \deg^{-1}(w^*)$
	is distinguishable with advantage at most $\epsilon / (10d)$. Thus, $\Hyb_1$ and $\Hyb_{2, \deg^{-1}(w^*)}$ are distinguishable with advantage at most $\epsilon / 10$.
	\begin{proof}
		This proof will be a modification of the proof in \cite{ishai2015public} for the simple case of weak extractible obfuscation.
		The key idea lies on two observations:
		\begin{enumerate}[labelsep=0.1em]
			\item We can go from $\Hyb_{2, j - 1}$ (or $\Hyb_1$) to a ``padded out'' version of $\Hyb_{2, j}$ by obfuscating a program which calls $\circNeigb$ internally and returns $\bot$ for the $j$-th input.
			\item We can go from $\Hyb_{2, j - 1}$ (or $\Hyb_1$) to a padded out version of itself.
			\item If an adversary can produce $\Hyb_{2, j - 1}$ (or $\Hyb_1$) and $\Hyb_{2, j}$ which are of the same size and can distinguish them with advantage at least $\epsilon / 10d$,
			then we can build an adversary, $\adversB$, which can produce a label $\phi(K, h)$ for $h \in H$ in $\Hyb_0/ \Hyb_1$.
		\end{enumerate}

		First, assume towards contradiction that there exists an adversary $\advers$ that can distinguish two consecutive hybrids
		in $O(T')$ time
		with polynomial advantage $\epsilon' > \epsilon / 10d$.
		% Following the proof sketch in \cite{ishai2015public} say that the input size to $\circNeigb$ is $n$.
		For simplicity, say that the input size to all of our circuits is $n$.
		Also, let $C_0$ be the circuit from the first hybrid and $C_1$ the one from the second.
		Let $\circMid_i$ be a circuit such that $\circMid_i(X) = C_0(X)$ if $X_i = 0$ and $\circMid_i(X) = C_1(X)$ if $X_i = 1$.
		Note that by our construction of the hybrids, $C_0$ and $C_1$ differ on at most 1 input (which is the appended vertex $y$ to $Y^*$);
		call this input $\alpha$.
		Then, $\circMid_i = C_0$ if $\alpha_i = 0$ and $\circMid_i = C_1$ if $\alpha_i = 1$.
		So, if we build an adversary $\adversB$ to tell if $\circMid_i = C_0$ or $\circMid_i = C_1$ with probability $\gamma$,
		we have that $\adversB(C_0, C_1)$ can tell if $\alpha_i$ is $0$ or $1$ with probability $\gamma$.
		Thus, $\adversB$ can reconstruct $\alpha$ with probability at least $\gamma^n$.
		Note that this implies that $\adversB$ can learn $\phi(K, y)$ where $y \in H$ (the set of hard-to-guess vertices) by construction.
		But here we run into a problem, the game the distinguishing adversary plays between $C_0$ and $C_1$ (and so forth) does not give the circuits from \emph{both} games to $\advers$.
		Rather, $\advers$ gets either the circuit $C_0$ or $C_1$.
		Note though that given $C_0$, $\advers$ can construct a larger version of $C_1$ by obfuscating a program which calls $C_0$ internally and returns $\bot$ for the $j$-th input.
		Call this larger circuit $C_1'$ with circuit size $\lambda'$.
		Moreover, $\advers$ can construct a larger version of $C_0$, which we will call $C_0'$, by simply padding it out to the size of $C_1'$.
		Now, if there exists a distinguishing adversary for this larger game with circuit size $\lambda'$, we can break the inductive hypothesis of the scheme with circuit size $\lambda$.

		

		\LS{Note that we are showing that if an adversary can distinguish between hybrids with a larger parameter size, then it can break the security of lower parameter hybrids. So, we then need to apply the same argument to the smaller parameter sized versions to update each recursive step. This basically means that we either only support poly-log depth (if we have multiplicative overhead iO) or we assume existance of additive overhead iO}
		% https://eprint.iacr.org/2015/1023.pdf
		Now, if $\adversB$ can distinguish $C_0, C_1$, we can create and adversary $\adversD$
		which can output a label of a vertex in $H$ with probability $\gamma^n$.
		$\adversD(C_0)$ proceeds as follows:
		\begin{enumerate}
			% TODO: this may not be the correct specification
			\item Create a circuit $\iO(C_1')$ which calls $C_0$ internally but returns $\bot$ instead of the $j$-th output
			if $v = y$. Also, create $\iO(C_0')$ to be the padded version of $C_0$ such that $\iO(C_0')$ and $\iO(C_1')$
			are the same size.
			\item Run $\adversB(\iO(C_0'), \iO(C_1'))$ $n$ times to get the differing input
			string $m \in \set{0, 1}^n$. Return $m$.
			% TODO: urr sizes do not match up vis a vis padding
			% TODO: may be underspecified hybrid...
		\end{enumerate}

		% As $C_1$ can be created by an adversary with access to $C_0$, an adversary, $\adversD$, could then
		% produce a label for a vertex in $H$ via simulating calls to $\adversB$.
		% and thus gives our desired contradiction.
		% Moreover, we have that $\advers(C_0)$ (where $C_0 = \circNeigb$) can construct $C_1$ (and thus $\circMid_i$)
		% by obfuscating a program which calls $\circNeigb$ internally and returns $\bot$ for the $j$-th input
		% if the input vertex is $u$ such that $(j, u) \in Y^*$.

		So now, we just need to build $\adversB$ to tell if $\circMid_i = C_0$ or $C_1$ with probability $\gamma^n \geq \epsilon$.
		To do so, we make oracle calls to $\advers$:

		% Then, $\advers$ can distinguish between $C^M$ via the following:
		\begin{enumerate}
			\item Run $I =\left\lceil 
			\frac{12\left(\ln 2 + \ln n - \ln\left(1 - \eps\right) + \ln 2\right)}{\eps'} \right\rceil$ iterations of the following experiment to estimate advantage $\epsilon'_b$ for $b \in \set{0, 1}$
				\begin{enumerate}
					\item Sample a random obfuscation of $C_b$ via re-obfuscating the existing $C_b$
					\item Sample a random obfuscation of $\circMid_i$ via re-obfuscating $\circMid_i$
					\item Have $\advers$ distinguish between $C_b$ and $\circMid$
					\item Output 1 if successful.
				\end{enumerate}
			Note that we can estimate $\epsilon'_b$ as the number of successful runs, which we will denote $\sum_{j \in [I]} S_{i, j}$, divided by $I$.
			\item If $\epsilon'_1 > \epsilon'_0$, then $\circMid = C_0$, otherwise, $\circMid = C_1$.
		\end{enumerate}

		Note that $\adversB$ runs in time $O(T' I)$. So, if we set the upper-bound on the runtime of the adversary in \cref{eq:guessPhi}
		to $O(T' I))$, then $\adversB$ can learn $\phi(K, y)$ with probability $\gamma^n \geq \frac{\epsilon}{10d}$.

		We differ the proof that $I$ is the correct choice of parameters such that
		$\gamma^n \geq \frac{\epsilon}{10d}$ to \cref{appendix:paramHybB}.
		% TODO: URRRR with \eps/10d???...
	\end{proof}
	
\end{lemma}

\begin{lemma}
	The game in $\Hyb_{2, \deg^{-1}(w^*)}$ is indistinguishable from $\Hyb_3$ with probability
	at most $\eps/10$.
	\begin{proof}
		As with \cref{lemma:hybA}, 
		the indistinguishably follows directly from the definition of indistinguishable obfuscation.
	\end{proof}
\end{lemma}

\begin{lemma}
	\label{lemma:hybC}
	% TODO: assuming this then that
	The game in $\Hyb_{3}$ is indistinguishable from $\Hyb_4$.
	\begin{proof}
		Assume towards contradiction that $\eps \in \poly(1/\lambda)$.
		We now show that if the advantage of $\advers$ is greater than $\eps/10$, then we can
		create a reduction, $\adversB$, which can break the security of the PRF at the punctured point.
		$\adversB$ first chooses a message $w^*$ and submits this to the constrained PRF challenger and gets back the punctured PRF key
		$K(\set{w^*})$ and challenge $a$. $\adversB$ then runs the experiment in $\Hyb_{2, \deg^{-1}(w^*)}$
		except that $z^* = f(a)$. If $a$ is the output of the PRF, then we are in $\Hyb_{2, \deg^{-1}(w^*)}$,
		if $a$ is the output of a random function, then we are in $\Hyb_3$.
	\end{proof}
\end{lemma}

\begin{algorithm}[H]
	\caption{
		Circuit for the neighbor function, $\circNeigb$ with PRF key
		$K$ and constant $w^*, z^*$%, y^*_1, y^*_2, \dots, y^*_d$
	}
	\begin{algorithmic}[1]
		\Function{$\circNeigb$}{$X, v$}
			\If{$v \neq w$ and $f(X) \neq f(F(K, v))$}
				\State \Return $\bot$
			\EndIf
			\If{$v = w$ and $f(X) \neq z^*$}
				\State \Return $\bot$ 
			\EndIf
		 	\If{$\Gamma(v) = \emptyset$}
				\State \Return $\bot$
			\EndIf
			
			% \If{$v = w$}
			% 	\State \Return $z^*_1, z^*_2, \dots, z^*_d$
			% \EndIf
			\State $u_1, \dots u_d = \Gamma(v)$
			\State \Return $F(K, u_1), F(K, u_2), \dots, F(K, u_d)$
		\EndFunction
	\end{algorithmic}
	\label{alg:neighbHyb1}
\end{algorithm}
% TODO: change w to w^*


\begin{algorithm}[H]
	\caption{
		Circuit for the neighbor function, $\circNeigb$ with punctured PRF key
		$K(S)$ and constant $w^*, Y^*, J^*, z^*$%, y^*_1, y^*_2, \dots, y^*_d$
	}
	\begin{algorithmic}[1]
		\Function{$\circNeigb$}{$X, v$}
			\If{$v \neq w$ and $f(X) \neq f(F(K, v))$}
				\State \Return $\bot$
			\EndIf
			\If{$v = w$ and $f(X) \neq z^*$}
				\State \Return $\bot$ 
			\EndIf
		 	\If{$\Gamma(v) = \emptyset$}
				\State \Return $\bot$
			\EndIf
			% \If{$v = w$}
			% 	\State \Return $z^*_1, z^*_2, \dots, z^*_d$
			% \EndIf
			\State $u_1, \dots u_d = \Gamma(v)$
			\While{$\exists j^* \in [d], (j^*, u_j) \in Y^*$}
			 	\State Set $F(K, u_{j^*}) = \bot$
			\EndWhile
			\State \Return $F(K, u_1), F(K, u_2), \dots, F(K, u_d)$
		\EndFunction
	\end{algorithmic}
	\label{alg:neighbHyb2}
\end{algorithm}