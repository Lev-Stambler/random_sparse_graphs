\newcommand{\MPK}{\text{MPK}}
\newcommand{\MSK}{\text{MSK}}
\newcommand{\SK}{\text{SK}}
\newcommand{\Enc}{\texttt{Enc}}
\newcommand{\Dec}{\texttt{Dec}}
\newcommand{\FEEnc}{\text{FE.Enc}}
\newcommand{\FEDec}{\text{FE.Dec}}
\newcommand{\FESetup}{\text{FE.Setup}}
\newcommand{\FEKeygen}{\text{FE.Keygen}}
\newcommand{\FECT}{\text{CT}}
\newcommand{\adversD}{\mathcal{D}}


\section{Preliminaries}
\subsection{Punctured PRF}
A punctured PRF is a simple type of constrained PRF (\cite{boneh2013constrained,boyle2014functional,kiayias2013delegatable})
where a PRF is well defined on all inputs except for a specified, polynomial-sized set.
We will adopt the notion specified in \cite{sahai2014use}.

\begin{definition}[Punctured PRF]
	A puncturable family of PRF	s $F$ mapping is given by a tuple of algorithms $(\text{Key}_F, \text{Puncture}_F, \text{Eval}_F)$.
	satisfying the following conditions: \begin{itemize}
		\item Functionality preserved under puncturing: For every PPT adversary $\advers$,
		$S \subseteq \set{0, 1}^{n}$
		and every $x \in \{0, 1\}^n$ where $x \notin S$, we have that
		\begin{equation*}
			\Pr\left[\texttt{Eval}_F(K, x) = \texttt{Eval}_F(K_S, x) \mid K \gets \texttt{Key}_F(1^\lambda), \texttt{K}_S = \texttt{Puncture}_F(K, S)\right] = 1.
		\end{equation*}
		\item Pseudorandom at punctured points: For every PPT adversary $\advers, \adversB$ such that
		$\advers(1^\lambda)$ outputs a set $S$ and state $\mathbf{st}$, consider an experiment where $K \gets \texttt{Key}_F(1^\lambda)$
		and $K_S = \texttt{Puncture}_F(K, S)$. Then, we have that
		\begin{equation*}
			\left|\Pr\left[\adversB(\texttt{st}, K_S, S, \texttt{Eval}_F(K, S)) = 1 \right]
				- \Pr\left[\adversB(\texttt{st}, K_S, S, U_{m \cdot |S|}) \right] \right| \leq \negl(\lambda).
		\end{equation*}
	\end{itemize}
\end{definition}

\subsection{Indistinguishable Obfuscation}
We will use the definition of indistinguishable obfuscation as presented in \cite{garg2016candidate}.
\begin{definition}[Indistinguishable obfuscation]
	A uniform PPT machine $\mathcal{O}$ is an indistinguishable obfuscator for a class of circuits $\mathcal{C}$ if for every circuit $C \in \mathcal{C}$
	we have that
	\begin{equation*}
		\Pr[C'(x) = C(x) \mid C' \gets \mathcal{O}(C)] \leq \negl(\lambda)
	\end{equation*}
	and for any PPT distinguisher $\adversD$ and two pairs of circuits $C_0, C_1$ such that
	$C_0(x) = C_1(x)$ for all $x$, then
	\begin{equation*}
		\bigg|\Pr\left[\adversD(\mathcal{O}(\lambda, C_0)) = 1\right] - \Pr\left[\adversD(\mathcal{O}(\lambda, C_1)) = 1\right]\bigg|.
	\end{equation*}
	
\end{definition}

\begin{definition}[Homomorphic Indistinguishable Obfuscationf (\cite{bhushan2023homomorphic})]
	We will use the definition of homomorphic indistinguishable obfuscation as presented in \cite{bhushan2023homomorphic}.
	Homomorphic indistinguishable obfuscation ($\HiO$) is a variation on indistinguishable obfuscation where an obfuscated circuit, $C$, can be composed
	with another circuit $C'$ to produce an obfuscated circuit $C \circ C'$ that computes $C(x) \circ C'(x)$ for all $x$.
	As outlined in \cite{bhushan2023homomorphic}, the size of the circuit remains polynomial after a polynomial number of compositions.
	Formally, an $\HiO$ scheme consists of the following three algorithms
	\begin{itemize}
		\item $\Obfuscate(1^\lambda, C)$: Takes as input a circuit $C$ and outputs an obfuscated circuit $\hat{C}$.
		\item $\Eval(\hat{C}, x)$: Takes as input an obfuscated circuit $\hat{C}$ and an input $x$ and outputs a string $y = C(x)$.
		\item $\Compose(\hat{C}, C')$: Takes as input an obfuscated circuit $\hat{C}$ and a circuit $C'$ and outputs an obfuscated circuit $\hat{C'}$ such that
		$\hat{C'}(x) = (C' \circ C)(x)$ for all $x$.
	\end{itemize}
	The scheme must satisfy standard notions of correctness and indistinguishably, though adopted to the homomorphic setting. Specifically, we require
	\begin{itemize}
		\item \textbf{Homomorphic Indistinguishablity}: For any $\lambda, k \geq 0$, and circuits $C_0^0, \dots, C_k^0$ and $C_0^1, \dots, C_k^1$, of size at most $k$ where
		\begin{equation*}
			C_k^0 \circ \cdots \circ C_0^0 = C_k^1 \circ \cdots \circ C_0^1,	
		\end{equation*}
		then it holds that
		\begin{align*}
			&\Compose(\cdots \Compose(\Obfuscate(1^\lambda, C_0^0), C_1^0), \cdots, C_k^0) \\
			\compInd \;\;
			&\Compose(\cdots \Compose(\Obfuscate(1^\lambda, C_0^1), C_1^1), \cdots, C_k^1).	
		\end{align*}
	\end{itemize}
\end{definition}