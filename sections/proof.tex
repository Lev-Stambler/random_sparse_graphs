\subsection{Proof of \cref{thm:labelExtract}}
Intuitively, showing how to construct an extractor given an adversary with advantage seems to be an
intractable problem. Rather than approach the proof in this way, we will show the contrapositive.
I.e.\ we will assume that there does not exist a uniform extractor for a fixed $\advers$ and graph instance and then show that
such an adversary has negligible advantage. We will do this by making an inductive argument on the graph and parameter $S$, finally giving us \cref{thm:labelExtract} for $S \geq O(\depth \cdot d)$.\\


We will first define some useful notation. Let $H_\inductiveInd \subset \Z^+ \times V$ be a ``hard'' set of 
composition depth and vertices for the $\inductiveInd$-th step of induction such that $\advers$ cannot, with non-negligible probability, produce
$\phi(K, h)$ where $(s, h) \in H_\inductiveInd$. More formally, for any fixed uniform $\adversV$,
\begin{equation*}
	\Pr\left[\adversV\left(\advers, \HiO(\circNeigbBase^s), v_0, h, \labelFunc(K, v_0)\right) = \phi(K, h)\right] \leq \negl(\lambda).
\end{equation*}
Further, let $\pathSet$ be the set of all paths from $v_0$ to $v$ in $\graph$.

% TODO: make base case proof more clear and intuitive
% Maybe add a picture
% Explain how we are just using this cut thing existentially
\subsubsection*{Inductive Proof}
Now, we will start with the base case. The base case is in essence non-cryptographic and holds for any $S \in \Z$.
\begin{lemma}[Base Case Game]
	\label{lemma:cutBaseCase}
	Assuming that there is no extractor $E$ such that 
	$$\Pr[E(\advers,\HiO(\circNeigb), v_0, v) = P] \geq \frac{1}{p(\lambda)}$$
	where $P \in \pathSet$, then for any PPT $\advers$, there exists some graph cut
	$C_E \subset E$ which separates $v_0$ and $v$ and a set $C_V$ such that
	\begin{equation}
		\label{eq:cutLabel}
		\Pr[\advers(\obfFN(\circNeigb), v_0, v, \phi(K, v_0)) \in \phi(K, C_V)] < \eps
	\end{equation}
	for any $\compDepth \geq \poly(\lambda)$.
	We have
	\begin{equation*}
		H_0 = \Z^+ \times \set{ u \mid (w, u) \in C_E \text{ and } u \in T} \bigcup \set{ w \mid (w, u) \in C_E \text{ and } w \in T}.
	\end{equation*}
	where $V = S \bigcup T$ is a partition of $V$ such that $v_0 \in S$ and $v \in T$.
	In words, for any composition depth, $H_0$ are the vertices just adjacent to the cut and on the same side as $v$.
	\begin{proof}
		We will show that if $\advers$ can break \cref{eq:cutLabel}, then we can construct an extractor,
		$E$, which finds a path from $v_0$ to $v$ with non-negligible probability.

		Assume that for every possible cut, $\advers$ is able to produce a single label in this cut for a vertex $w$.
		Then, we note that there must be at least 1 path from $v_0$ to $w$ and from $w$ to $v$ as otherwise, $w$ would not be in the cut.
		Moreover, we note that $\advers$ must be able to produce a label for all vertices on at least one path
		from $v_0$ to $w$ as otherwise, we can change the cut to include the edges between where
		$\advers$ is able to produce a label and not able to produce a label. Using the same argument,
		we can show that $\advers$ must be able to produce all labels on a path from $w$ to $v$.

		Note that $\advers$ is not given the specific cut $C_E$ but rather $C_E$ is chosen based off of the adversary.
		So, we can build an extractor to do the following:
		\begin{enumerate}
			\item Create an iO obfuscated circuit with a random key, $K'$, for $\circNeigb$ and create circuit $\obfFN(\circNeigb)$
			      as well as $\phi(K', v_0)$
			\item Run $\advers(\obfFN(\circNeigb), v_0, v, \phi(K', v_0))$ to get all labels $\phi(K', v_0), \dots \phi(K', v)$
			      for some path from $v_0$ to $v$.
			\item Recreate the path from $v_0$ to $v$ via checking which vertex matches to adjacent labels in the path:
			      % TODO: this assumes no cycles!!!
			      I.e.\ starting with $\ell = 0$, we can learn the $\ell + 1$ vertex via finding $j \in [d]$ such that
			      $\circNeigb(\phi(K', v_\ell), v_\ell)_j \in \set{\phi(K', v_0), \dots, \phi(K', v)}$
			      and then setting $v_{\ell + 1} = \Gamma(v_\ell)_j$.
		\end{enumerate}
	\end{proof}
\end{lemma}

We can look at \cref{lemma:cutBaseCase} as a ``base case'' of sorts. We now inductively build up a series of games
such that $\advers$ cannot find any label in $V_1$ where $V_1$ are the vertices on side of the cut (as defined in \cref{lemma:cutBaseCase})
which contain $v$.


\begin{lemma}[Inductive Game Hypothesis]
	Assuming adaptive security of constrained PRFs, one way functions, the existence of homomorphic indistinguishable obfuscation,
	and the hardness of producing a label for $h \in H_\inductiveInd$,
	then we have that
	$$H_{\inductiveInd + 1} = \left\{ H_\inductiveInd \cup \set{(s, \Gamma(v)) \mid (s, v) \in H_\inductiveInd} \right\} \bigcap \set{ [2 (\inductiveInd + 1) \cdot d, \infty) \times V}.$$

	\begin{proof}
		We are going to use a series of indistinguishable hybrids along with the circuit defined in \ref{alg:neighbHyb1} to show the above
		\begin{itemize}
			\item $\Hyb_0$: In the first hybrid, the following game is played
				\begin{enumerate}
					\item $K \gets \binSet^{\lambda'}$ and $\phi(K, v_0) = (F(K, v_0), v_0)$ where $K$ is some fixed secret drawn from a uniform distribution
					\item The challenger generates $\HiO(\circNeigb)$ and gives the program to $\advers$
					\item The challenger gives the adversary $h^*$ in plaintext.
					\item $\advers$ outputs guess $g$ and wins if $g = \phi(K, h^*)$ %hmmm... do we give w?
				\end{enumerate}
			
			\item $\Hyb_1$: We replace $\circNeigb$ with the punctured circuit defined in \cref{alg:neighbHyb1}.
			Fix the constant $z^* = f(F(K, w^*))$
			% and $y^*_1 = f(F(K, \Gamma(w^*)_1)), \dots, y^*_d = f(F(K, \Gamma(w^*)_d))$.
			\item $\Hyb_{2, 1}$
			We replace circuit~\ref{alg:neighbHyb1} with circuit~\ref{alg:neighbHyb2} where we 
			set $Y^* = (1, y)$ such that $\Gamma(y)_1 = h^*$. So then, we have that
			have $F(K, \Gamma(y)_1) = \bot$. Moreover, we set the punctured set, $S$ to $\emptyset$ (i.e.\ we do not puncture the PRF).
			\item $\Hyb_{2, j}$ for $j \in \set{2, \dots, \deg^{-1}(h^*)}$
			We replace $Y^*$ with $Y^* \cup (j, y)$ such that $\Gamma(y)_j = h^*$.
			Note after the last of these hybrids, we have that $F(K, h^*)$ is always set to $\bot$ on line \ref{alg:lineret} of circuit~\ref{alg:neighbHyb2}.
			\item $\Hyb_3$: We puncture the PRF at $h^*$ and set $S = \set{h^*}$.
			\item $\Hyb_4$: Set $z^* = f(t)$ where $t$ is chosen at random %TODO: specify field size
		\end{itemize}
		Finally, we can note that if $\Hyb_0 \compInd \Hyb_4$, then
		\begin{equation*}
			\Pr[\advers(\circNeigb, v_0, h^*, \phi(K, v_0)) = \phi(K, h^*)] 
			\compInd
			\Pr[\advers(\circNeigb^*, v_0, h^*, \phi(K, v_0)) = \phi(K, h^*)]
		\end{equation*}
		where $\circNeigb^*$ is the circuit which sets $z^*$ to be the image on a one way function of a randomly chosen point.
		As we will show in \cref{lemma:hybA}, \cref{lemma:hybB}, and \cref{lemma:hybC},
		an adversaries advantage between games in $\Hyb_0$ and $\Hyb_3$ is at most $\epsilon / 2$.
		Thus, if $\advers$ can produce $\phi(K, v) = (\sigma_v, v)$ with advantage $\epsilon / 2$
		in $\Hyb_3$, then $\advers$
		can find a pre-image for $z^*$ under $f$ with non-negligible probability and thus break the security of a one way function.
		We then have that the advantage of the adversary in $\Hyb_0$ cannot be more than $\epsilon$.
	\end{proof}
\end{lemma}


\begin{lemma}
	\label{lemma:hybA}
	% TODO: workout
	$\Hyb_0$ and $\Hyb_1$ are distinguishable with advantage at most $\epsilon / 10$.
	\begin{proof}
		Assume towards contradiction that $\eps \in \poly(1/\lambda)$.
		Note that for all inputs $(z, v)$ to $\circNeigb$ as defined in circuit~\ref{alg:neighb} and circuit~\ref{alg:neighbHyb1}
		are equivalent and thus indistinguishable by the definition of indistinguishable obfuscation.
		So, if $\eps \in \poly(\lambda)$, then an adversary cannot distinguish the hybrids with probability more than $\epsilon / 10$.
	\end{proof}
\end{lemma}

\begin{lemma}
	\label{lemma:hybB}
	Each hybrid from $\Hyb_1$ to $\Hyb_{2, 1}$ and $\Hyb_{2, j - 1}$ to $\Hyb_{2, j}$ for $j \in 2, \dots, \deg^{-1}(h^*)$
	is distinguishable with advantage at most $\epsilon / (10d)$. Thus, $\Hyb_1$ and $\Hyb_{2, \deg^{-1}(w^*)}$ are distinguishable with advantage at most $\epsilon / 10$.
	\begin{proof}
		This proof will be a modification of the proof in \cite{ishai2015public} for the simple case of weak extractible obfuscation.
		The key idea lies on two observations:
		\begin{enumerate}[labelsep=0.1em]
			% TODO: explitly define these circuits in the construction so we can explicity define circ pad's size
			\item We can go from $\Hyb_{2, j - 1}$ (or $\Hyb_1$) to a ``padded out'' version of $\Hyb_{2, j}$ by 
			composing $\circNeigb$ with a circuit which returns $\bot$ for the $j$-th input.
			\item We can go from $\Hyb_{2, j - 1}$ (or $\Hyb_1$) to a re-randomized, slightly larger, version of itself by composing $\HiO(\circNeigb)$ with $C_{iden}$ (the identity circuit).
			\item If an adversary can produce $\Hyb_{2, j}$ solely from $\Hyb_{2, j - 1}$ (or $\Hyb_1$) and can distinguish them with advantage at least $\epsilon / 10d$,
			then we can build an adversary, $\adversB$, which can produce a label $\phi(K, h)$ for $h \in H$ in $\Hyb_0/ \Hyb_1$.
		\end{enumerate}

		For simplicity, say that the input size to all of our circuits is $n$. Let $s \in \Z$ such that $s \geq 2 \cdot d \cdot \inductiveInd$.
		Also, let $C_0^s$ be the circuit from the first hybrid and $C_1^s$ the one from the second.

		At a high level, we will show that an adversary can create a \emph{larger version} of $C_1$ given only $C_0$. Then, 
		if an adversary can distinguish between $C_0^{s + 2}$ and $C_1^{s + 2}$, the adversary can recover a label in $H_\inductiveInd$
		for instances of $C_0^s$.

		First, given $C_0^s$, $\advers$ can construct a larger version of $C_1^{s + 1}$ by 
		composing $C_0^s$ with a program that returns $\bot$ for the $j$-th input of $h^*$ on line \ref{alg:lineret}.
		$\advers$ can also produces $C_0^{s + 1}$ from $C_0^s$ via composing $C_0^s$ with the identity circuit.

		% Moreover, $\advers$ can construct a larger version of $C_0$, which we will call $C_0'$, by simply padding it out to the size of $C_1'$.
		% Now, if there exists a distinguishing adversary for this larger game with circuit size $\lambda'$, we can break the inductive hypothesis of the scheme with circuit size $\lambda$.
		Now, assume towards contradiction that there exists an adversary $\advers$ that can distinguish $C_0^{s + 2}$ and $C_1^{s + 2}$ with advantage $\epsilon' > \epsilon / 10d$
		in $O(T')$ time.
		Let $\circMid_i$ be a circuit such that $\circMid_i(X) = C_0^{s + 2}(X)$ if $X_i = 0$ and $\circMid_i(X) = C_1^{s + 2}(X)$ if $X_i = 1$.\\
		% TODO: composition indisting?
		We can see that $C_0^{s + 1}$ and $C_1^{s + 1}$ differ on at most 1 input which we will call $\alpha$. %(which is the appended vertex $y$ to $Y^*$);
		% call this input $\alpha$.
		Then, $\circMid_i = C_0^{s + 2}$ if $\alpha_i = 0$ and $\circMid_i = C_1^{s + 2}$ if $\alpha_i = 1$.
		So, if we build an adversary $\adversB$ which can tell if $\circMid_i = C_0^{s + 2}$ or $\circMid_i = C_1^{s + 2}$ with probability $\gamma$,
		we have that $\adversB(C_0^{s + 2}, C_1^{s + 2})$ can be used to check if $\alpha_i$ is $0$ or $1$ with probability $\gamma$.
		So then, $\adversB$ can be used to learn $\phi(K, \alpha)$ with probability at least $\gamma^n$.
		Because $(s, \alpha) \in H$, we get that $\adversB$ can break the inductive hypothesis!\\
		% TODO: H should m\paramPad(\inductiveInd, \lambda)aybe be a s \cross v
		\linebreak
		% 
		To build $\adversB$ to tell if $\circMid_i = C_0$ or $C_1$ with probability $\gamma^n \geq \epsilon$, we will make oracle calls to $\advers$:
		\begin{enumerate}
			\item Run $I =\left\lceil 
			\frac{12\left(\ln 2 + \ln n - \ln\left(1 - \eps\right) + \ln 2\right)}{\eps'} \right\rceil$ iterations of the following experiment to estimate advantage $\epsilon'_b$ for $b \in \set{0, 1}$
				\begin{enumerate}
					\item Sample an obfuscation of $C_b^{s + 2}$ by composing $C_b^{s + 1}$ with the identity circuit.
					% TODO: define this switching function a bit better. But it simply decides between saying \bot or not
					\item Sample an obfuscation of $\circMid_i$ by composing a function which will return $\bot$ on line~\ref{alg:lineret} if $\alpha_i = 0$ and otherwise return the output of $C_0^{s + 1}$.
					\item Have $\advers$ distinguish between $C_b^{s + 2}$ and $\circMid_i$
					\item Output 1 if successful.
				\end{enumerate}
			Note that we can estimate $\epsilon'_b$ as the number of successful runs, which we will denote $\sum_{j \in [I]} \success_{i, j}$, divided by $I$.
			\item If $\epsilon'_1 > \epsilon'_0$, then $\circMid(x) = C_0^{s+2}(x)$, otherwise, $\circMid = C_1^{s+2}(x)$.
		\end{enumerate}

		Note that $\adversB$ runs in time $O(T' I)$. So, if we set the upper-bound on the runtime of the adversary in \cref{eq:guessPhi}
		to $O(T' I))$, then $\adversB$ can learn $\phi(K, y)$ with probability $\gamma^n \geq \frac{\epsilon}{10d}$.

		We differ the proof that $I$ is the correct choice of parameter to \cref{appendix:paramHybB}.
	\end{proof}
	
\end{lemma}

\begin{lemma}
	The game in $\Hyb_{2, \deg^{-1}(w^*)}$ is indistinguishable from $\Hyb_3$ with probability
	at most $\eps/10$.
	\begin{proof}
		As with \cref{lemma:hybA}, 
		the indistinguishably follows directly from the definition of indistinguishable obfuscation.
	\end{proof}
\end{lemma}

\begin{lemma}
	\label{lemma:hybC}
	% TODO: assuming this then that
	The game in $\Hyb_{3}$ is indistinguishable from $\Hyb_4$.
	\begin{proof}
		Assume towards contradiction that $\eps \in \poly(1/\lambda)$.
		We now show that if the advantage of $\advers$ is greater than $\eps/10$, then we can
		create a reduction, $\adversB$, which can break the security of the PRF at the punctured point.
		$\adversB$ first chooses a message $w^*$ and submits this to the constrained PRF challenger and gets back the punctured PRF key
		$K(\set{w^*})$ and challenge $a$. $\adversB$ then runs the experiment in $\Hyb_{2, \deg^{-1}(w^*)}$
		except that $z^* = f(a)$. If $a$ is the output of the PRF, then we are in $\Hyb_{2, \deg^{-1}(w^*)}$,
		if $a$ is the output of a random function, then we are in $\Hyb_3$.
	\end{proof}
\end{lemma}

