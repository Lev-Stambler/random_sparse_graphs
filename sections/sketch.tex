\section{A sketch for the boys}
%%%%%%%%%%%% Commands %%%%%%%%%%%%
\newcommand{\graph}{\mathcal{G}}
\newcommand{\verts}{V}
\newcommand{\edges}{E}
\newcommand{\maxDeg}{d}
\newcommand{\nVerts}{n}
\newcommand{\embedFn}{\phi}
\newcommand{\idealEmbedFn}{\Phi}
\newcommand{\circNeigb}{C_\Gamma}

\newcommand{\advers}{\mathcal{A}}

\newcommand{\minEntropy}{I_\texttt{min}}
%%%%%%%%%%%% End Commands %%%%%%%%%%%%

\subsection{Graph Label Randomization}
Say that we have a sparse graph $\graph = (\verts, \edges)$ such that $|\verts| = n$
and $\forall v \in \verts, \deg(v) \leq d$.

We want to ``randomize'' the labels of the graph via a poly-time embedding function $\phi$
such that the embeddings are indistinguishable from a truly random embedding, $\idealEmbedFn$.

We model $\idealEmbedFn$ as function from $\verts$ to $\{0, 1\}^{c \cdot \lambda}$ for some small constant $c$
such that 
\begin{equation*}
	\minEntropy(\phi(V) \mid V = v) \geq 2 \cdot \lambda.
\end{equation*}
Indeed, we do not require that the labels are uniformly random, but rather that each label is ``random enough'',
containing at least $2 \lambda$ bits of min-entropy.


We can now propose a game to characterize the pseudo-random embedding $\phi$.
For any PPT adversary, $\advers$,
\begin{equation}
	\big|
		\Pr\left[\advers(\phi(v_1), \dots \phi(v_{i})) = 1\right]
		 - 
		\Pr\left[\advers(\idealEmbedFn(v_1), \dots \idealEmbedFn(v_{i})) = 1\right]
	\big|
		\leq \negl(\lambda)
\end{equation}
for some $i \in \poly(\lambda)$.

The above game may prove to be uninteresting as we can simply describe $\phi$ to be a PRF 
which takes in the vertex label and outputs a pseudo-random string of length $c \cdot \lambda$.

This brings us to our notion of graph-label randomization obfuscation (GRO).

\begin{definition}[Graph-label randomization obfuscation (GRO, pronounced grow)]
	Given a circuit $\circNeigb$ realizing $\phi \circ \Gamma \circ \phi^{-1}: \set{0, 1}^{c \cdot \lambda} \rightarrow \set{0, 1}^{c \cdot d \cdot \lambda}$ (the neighbor function for the embedded space)
	and any polynomial time adversary,
	\begin{equation}
	\big|
		\Pr\left[\advers(\phi(v_1), \dots \phi(v_{i}), \circNeigb) = 1\right]
		 - 
		\Pr\left[\advers(\idealEmbedFn(v_1), \dots \idealEmbedFn(v_{i}), \circNeigb) = 1\right]
	\big|
		\leq \negl(\lambda)
	\end{equation}
\end{definition}

