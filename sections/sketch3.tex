\section{A sketch for the boys}
%%%%%%%%%%%% Commands %%%%%%%%%%%%
\newcommand{\graph}{\mathcal{G}}
\newcommand{\verts}{V}
\newcommand{\inner}{\texttt{inner}}
\newcommand{\edges}{E}
\newcommand{\maxDeg}{d}
\newcommand{\nVerts}{n}
\newcommand{\embedFn}{\phi}
\newcommand{\idealEmbedFn}{\Phi}
\newcommand{\circNeigb}{C_\Gamma}
\newcommand{\PRF}{\texttt{PRF}}
\newcommand{\Hyb}{\texttt{Hyb}}
\newcommand{\adversB}{\mathcal{B}}

\newcommand{\minEntropy}{I_\texttt{min}}
%%%%%%%%%%%% End Commands %%%%%%%%%%%%

\subsection{Graph Label Randomization in ROM}
Say that we have a sparse graph $\graph = (\verts, \edges)$ such that $|\verts| = n$
and $\forall v \in \verts, \deg(v) = d$. (TODO: padding).

Then, we want to create a pseudo-randomized label mapping of the graph, $\embedFn: \set{0, 1}^\lambda \times \verts \rightarrow \{0, 1\}^{c \cdot \lambda}$
such that $\embedFn$ is deterministic and pseudo-random. In particular, we require that for an adversary 
that does not know a path from $v$ to $u \in \set{v_1, ..., v_p}$ or $u$ to $v$ where $v \neq u$, then for $K \randomGet \set{0, 1}^\lambda$,
\begin{align}
	\Pr\left[\advers(v, \embedFn(K, v_1), \dots \embedFn(K, v_{p}), v_1, \dots, v_p, \circNeigb) = \embedFn(K, v)\right]
	\leq \negl(\lambda) 
	\label{eq:gameGRO}
\end{align}
% \begin{align}
% 	&\Pr\left[\advers(\embedFn(K, v), \embedFn(K, v_1), \dots \embedFn(K, v_{p}), v_1, \dots, v_p, \circNeigb) = 1\right]
% 	\label{eq:gameGRO}
% 	\\
% 	&- \Pr\left[\advers(\embedFn(K', v), \embedFn(K, v_1), \dots \embedFn(K, v_{p}), v_1, \dots, v_p, \circNeigb) = 1\right]
% 		\leq \negl(\lambda) \notag
% \end{align}
where $\circNeigb$ is the neighbor function for the embedded space: i.e.\ $\circNeigb = \embedFn \circ \Gamma \circ \embedFn^{-1}$.

\subsection{The Construction}
\begin{algorithm}[H]
	\caption{
		The circuit for the neighbor function, $\circNeigb$.
	}
	\begin{algorithmic}[1]
		\Function{$\inner_i$}{$\Dec(\phi(v)) = v, K$}
			\State $u_1, \dots, u_d = \Gamma(v)$
			\State $u = u_i$
			\State $r = H(K, u)$
			\State \Return $\FEEnc(\MPK, (u, K))$ where we encrypt with randomness from $r$.
			% \State $\Dec(\SK_{\circNeigb'}, \phi(v))$
		\EndFunction
		\Function{$\circNeigb$}{$\phi(v)$}
		 	\For{$i \in [d]$}
				\State $u_i = \FEDec(\SK_{\texttt{inner}_i}, \phi(v))$
			\EndFor
			\State \Return $(u_1, \dots, u_d)$
		\EndFunction
	\end{algorithmic}
	\label{alg:neighb}
\end{algorithm}

\begin{claim} 
	\label{claim:sec}
	\cref{eq:gameGRO} holds for a given vertex $v$ for any PPT adversary, $\adversB$ when $\circNeigb$ is implemented as in \cref{alg:neighb}.
\end{claim}
In order to prove the above, we must show that an adversary which does not know a path from $v$ to $v_1, \dots v_p$
essentially ``learns'' nothing of $K$.

\begin{lemma}[Does not learn $K$]
	\label{lemma:hardLearnK}
	For any PPT adversary $\advers$
	\begin{align}
		\label{eq:Kindist}
		\Pr[\advers(\phi(K, v_1), \dots, \phi(K, v_p), v_1, \dots, v_p, \circNeigb) = K]
			\leq \negl(\lambda). \notag
	\end{align}
	% \begin{align}
	% 	\big| &\Pr[\advers(\phi(K, v_1), \dots, \phi(K, v_p), v_1, \dots, v_p, \circNeigb) = 0]
	% 	\label{eq:Kindist}
	% 	\\
	% 	& - \Pr[\advers(\phi(K', v_1), \dots, \phi(K', v_p), v_1, \dots, v_p, \circNeigb) = 0]\big|
	% 		\leq \negl(\lambda). \notag
	% \end{align}
	\begin{proof}
		We proceed via a series of hybrids to build two computationally indistinguishable distributions.
		We then show that assuming $\advers$ can output $K$ with non-negligible probability, we can distinguish the two distributions.
		First, let $K' \randomGet \set{0, 1}^\lambda$.
		
		Then, we have the following hybrids
		\begin{itemize}
			\item $\Hyb_0$: The adversary plays the game outlined in \cref{alg:FEExprReal} where the circuits are $\inner_1, \dots, \inner_d$
			and encryptions of $(K, v_\ell)$ are done with true randomness
			\item $\Hyb_1$: As the above except that we replace the real protocol with the simulated one, \cref{alg:FEExprSim}
			\item $\Hyb_2$: As the above except that we invoke the ROM to replace $r$ with fixed randomness for each $u$
			% TODO: note that we cannot have this be the case... as randomness must be true for the simulator
			\item $\Hyb_3$: As the above except that we replace $\inner_i$ with $\inner_i'$ where $\inner_i'$ invokes the FE simulator, $\Sim$ when generating its output.
			Note that $\Sim$ must know the access pattern of $\advers$ to $\inner_i$ in order to simulate the output of $\inner_i$.
			So, we replace $\inner_i'$ working backwards from the last call to the simulator to the first.
			As the simulator is stateful, we can see that the simulator knows the access pattern.
			\item $\Hyb_4$: As the above except that we replace $K$ with $K'$.
			We can see that this is valid as both inputs to $\inner'_i$ and outputs of $\inner'_i$ are simulated and thus independent of $K$.
		\end{itemize}
	Now, we show that we can replace the true randomness used with randomness derived from a random oracle%PRF generated randomness
	\begin{itemize}
		\item $\Hyb_5^a$: As $\Hyb_4$ except that we fix any randomness used by the simulator to be $H(K', \cdot)$ generated randomness.
		% Note that $\PRF(K, \cdot)$ is independent of $K'$ and thus the underlying cipher-texts known to the adversary.
		\item $\Hyb_6^a$: As the above except that we replace $\inner'_i$ with $\inner_i$.
		\item $\Hyb_7^a$: As the above except that we replace the simulated version with the real protocol, except that this time we are working over randomness generated by $K'$
		% \item $\Hyb_8^a$: As the above except that we replace the true randomness with $\PRF(K, \cdot)$ generated randomness.
		% Note that we can do this as the cipher-texts are independent of $K$.
	\end{itemize}
	We now do something similar to the above except we swap $K$ and $K'$
	\begin{itemize}
		\item $\Hyb_5^b$: As $\Hyb_0$
		\item $\Hyb_6^b$: As the above except that we replace the true randomness with $H(K, \cdot)$ generated randomness.
		% Note that we can do this as the cipher-texts are independent of $K'$.
	\end{itemize}
	We now have that $\Hyb_8^a \approx \Hyb_6^b$.
	Assume that $\advers$ can output $K$ with non-negligible probability. Then, we can build $\advers'$ to distinguish
	$\Hyb_8^a$ and $\Hyb_6^b$.
	$\advers'$ proceeds as follows:
	\begin{enumerate}
		\item Call $\advers(\cdot)$ to get $K$
		% Hmmm not for directed graph
		\item Check whether $\circNeigb(\phi(K, v_\ell)) = \circNeigb(\FEEnc(K, \Gamma(v_\ell)))$ 
		If the above equality holds, output $1$, otherwise, output $0$.
	\end{enumerate}
	Assume that with probability $\alpha > \negl(\lambda)$, $\advers$ gives the correct $K$.
	Then, because $K'$ is random, we have that with high probability $\circNeigb(\phi(K, u_i)) \neq \circNeigb(\phi(K', u_i))$.
	Thus, $\advers$ can distinguish the two hybrids.
	\end{proof}
\end{lemma}

Now we can prove \cref{claim:sec},
\begin{proof}[Proof of \cref{claim:sec}]
First note that the adversary cannot learn $\phi(K, v)$ via calling $\FEEnc(K, v)$ as
that would imply the ability to construct an adversary that can output $K$, breaking \cref{lemma:hardLearnK}.

Thus, $\adversB$ can only learn $\phi(K, v)$ via calling $\circNeigb$ or manipulating given cipher texts.
We now proceed to show that this is computationally infeasible via a hybrid algorithm.

\begin{itemize}
	\item $\Hyb_0$: The adversary plays the game outlined in \cref{alg:FEExprReal} where the circuits are $\inner_1, \dots, \inner_d$
	and $\advers$ is given $\phi(K, v_1) = \FEEnc(v_1, K), \dots, \phi(K, v_p) = \FEEnc(v_p, K)$ where encryption randomness is derived from $\PRF(K, v_1), \dots, \PRF(K, v_p)$.
	% TODO: can we do this??? (I think that we have to go through this lengthy simulator argument we have above)
	\item $\Hyb_1$: As the above except that we invoke the random oracle to replace the encryption randomness with true random strings fixed for each $v_\ell$ where $\ell \in [p]$.
	% TODO: adjust above to get exponential hardness when we write negl
	% TODO: HMmmmmmmm... randomness
	\item $\Hyb_2$: As the above except that we replace the real protocol with the simulated one, \cref{alg:FEExprSim}
	\item $\Hyb_3$: As the above except that we replace $\inner_i(\phi(K, u))$ with $\bot$ if $\inner_i(\phi(K, u)) = \phi(K, v)$.
	Note that $\advers$ cannot distinguish between this and the above hybrid
	because $\advers$ does not know the path from $v_\ell$ to $v$
	and can thus not find $\phi(K, u)$ from repeated queries of the embedded neighbor function, $\circNeigb$
	nor can $\advers$ generate $\phi(K, u)$ as $K$ is hard to learn.
\end{itemize}

Indeed the simulated labels in $\Hyb_3$, which we will denote $\phi'(K, v_\ell)$, are simulated independently of $\phi(K, v)$.
So,
\begin{equation*}
		\Pr[\adversB\big(v, \phi'(K, v_1), \dots, \phi'(K, v_p), v_1, \dots, v_p, \circNeigb\big) = \phi(K, v)]
			 - \Pr[\adversB(v, \circNeigb) = \phi(K, v)]
	\leq \negl(\lambda).
\end{equation*}

We can then assert that
\begin{equation*}
		\Pr[\adversB\big(v, \phi(K, v_1), \dots, \phi(K, v_p), v_1, \dots, v_p, \circNeigb\big) = \phi(K, v)]
			 - \Pr[\adversB(v, \circNeigb) = \phi(K, v)]
	\leq \negl(\lambda).
\end{equation*}
% To prove the above, assume that we have an adversary $\adversB$ which has non-negligible advantage
% over $\Pr[\adversB(v, \circNeigb) = \phi(K, v)]$.
% TODO: we can play a game to distinguish the above, but we need to have an extra H(\phi(K, v)) to check 
% for correctness... this makes the proof tricky though because now we need to deal with H(\phi(K, v))
% the proof should be easy in ROM though


And, by \cref{lemma:hardLearnK}, we have that
\begin{equation*}
	\Pr[\adversB(v, \circNeigb) = \phi(K, v)] \leq \negl(\lambda)
\end{equation*}
thus concluding the proof.
% TODO: show the high probability part!!
\end{proof}

\section{Welded Tree Graph Construction}
\begin{algorithm}[H]
	\caption{
		The circuit for the neighbor function, $\circNeigb$.
	}
	\begin{algorithmic}[1]
		\Function{$\inner_i$}{$\Dec(\phi(v)) = v, K, K'$}
		 	\State $K_\alpha, K_\beta, K_\gamma = PRG(K')$
			% TODO: if coll
			\State $u_1, \dots, u_3 = \Gamma(v)$
			\State $a_1, a_2, a_3 = PRP(K_\gamma, 1), PRP(K_\gamma, 2), PRP(K_\gamma, 3)$ where we are permuting over $\set{1, 2, 3}$
			\State $u = u_{a_i}$
			\State $r = H(K, u)$
			\State \Return $\FEEnc(\MPK, (u, K, K'))$ where we encrypt with randomness from $r$.
			% \State $\Dec(\SK_{\circNeigb'}, \phi(v))$
		\EndFunction
		\Function{$\circNeigb$}{$\phi(v)$}
		 	\For{$i \in [d]$}
				\State $u_i = \FEDec(\SK_{\texttt{inner}_i}, \phi(v))$
			\EndFor
			\State \Return $(u_1, \dots, u_d)$
		\EndFunction
	\end{algorithmic}
	% \label{alg:neighb}
\end{algorithm}


% https://eprint.iacr.org/2016/361.pdf
% Maybe this defn??